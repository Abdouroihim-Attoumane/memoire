\documentclass[a4paper,12pt]{report}                % Format standard pour les mémoires
\usepackage{packages, macros}                       % Chargement des packages et macros

\begin{document}
% Page de garde
\begin{titlepage}
    \newgeometry{top=0cm, bottom=0cm, left=2cm, right=2cm} % Marges spécifiques à la page de garde
    \begin{tikzpicture}[remember picture, overlay]
        \fill[bleufonce] ([xshift=1cm]current page.north west) rectangle ([xshift=2cm,yshift=-6cm]current page.north west); % Rectangle en haut à gauche
        \fill[blue] (current page.north east) -- ([xshift=-1cm]current page.north east) -- ([yshift=-10cm]current page.north east) -- cycle; % Triangle en haut à droite
        \fill[blue] ([xshift=1cm] current page.south west) rectangle ([xshift=2cm,yshift=12cm]current page.south west); % Rectangle en bas à gauche
        \fill[bleufonce] (current page.south east) -- ([xshift=-1cm]current page.south east) -- ([yshift=10cm]current page.south east) -- cycle; % Triangle en bas à droite
    \end{tikzpicture}
    \begin{center}
        \begin{minipage}{0.95\textwidth}
            \includegraphics[width=7cm]{images/logo_fst_1.png}
            \hfill
            \includegraphics[width=3cm]{images/logo_ubs.png}
        \end{minipage}\\[0.5cm]
        \includegraphics[width=6cm]{images/logo_campus.png}\\[0.5cm]
        \HRule \\[0.5cm]
        \textcolor{bleufonce}{\textbf{\textsc{\LARGE Memoire de Recherche de II CYCLE}}} \\[1.5cm]
        \begin{tcolorbox}[
            enhanced,                   % active les fonctionnalités avancées
            colback=blue,
            colframe=blue,
            arc=3mm,
            boxrule=0mm,
            width=18cm,
            fontupper=\color{white}, 
            center title, 
            drop shadow={black!50!white, shadow xshift=2pt, shadow yshift=--2pt,opacity=0.6}
            ]
            \centering
            \textbf{\LARGE Analyse de l'impact des dernières réformes du baccalauréat au Sénégal sur le taux de réussite au Bac et l’insertion universitaire}
        \end{tcolorbox}
        \vspace{1cm}
        \textbf{\textit{\large Présenté et soutenu par:}}\\[0.5cm]
        \textcolor{blue}{\textbf{\LARGE ABDOUROIHIM Attoumane}}
    \end{center}
    \vspace{1cm}
    \begin{flushleft}
        \hspace{0.1cm} 
        \textmd{\large Pour l'obtention du double diplôme de \textbf{\Large Master 2},}
        \begin{center}
            \textmd{\Large \textcolor{blue}{\textbf{D'Ingénierie Mathématique et Numérique}}}
        \end{center}
        \vspace{0.5cm}
        \hspace{0.1cm}
        \textmd{\large Sous la direction de : \textbf{\Large Pr Bamba GUEYE \& Pr Mountaga LAM}}\\  
        \vspace{1cm}
        \hspace{1cm}
        \textit{\large Présenté le 5 juillet 2025, devant le jury composé de :}\\
        \vspace{0.5cm}
        \begin{center}
        \hspace{3cm}
        \begin{minipage}{0.3\textwidth}
            \textmd{\Large Dr Souley KANE }\\[0.3cm]
            \textmd{\Large Pr Papa NGOM}\\[0.3cm]
            \textmd{\Large Dr Aissatou SOW}
        \end{minipage}
        \hfill
        \begin{minipage}{0.4\textwidth}
            \textmd{\large Maître de conférence}\\[0.3cm]
            \textmd{\large Professuer Titulaire (President)}\\[0.3cm]
            \textmd{\large Maître de conférence}
        \end{minipage}
    \end{center}
    \end{flushleft}
    
\end{titlepage}
    \restoregeometry % Restauration des marges globales
    % Dédicace
    \begin{center}
        \section*{\textsc{\LARGE Dédicace}}
    \end{center}
    \bigskip
    \textbf{Je dédie ce travail à ma famille :}

    \bigskip

    À mes parents, M. \textbf{ATTOUMANE Abdallah} et Mme \textbf{ZALIHATAH Abdallah}, qui ont été un soutien inconditionnel tout au long de mon parcours académique, 
    tant sur le plan financier que moral. Merci pour vos conseils dans mes moments de doute et pour votre présence constante dans tous mes choix et dans toute ma vie.

    \bigskip

    À ma femme, Mme \textbf{ADELIA Soiffaouiddine}, dont la présence à mes côtés a été essentielle pour affronter cette vie ensemble. 
    Sans elle, ce travail n’aurait tout simplement pas existé.

    \bigskip

    À mes deux grandes sœurs, Mme \textbf{ITFAOU Attoumane} et Mme \textbf{ROIHMA Attoumane}, pour leur affection et leur soutien inconditionnel.

    \bigskip

    À ma belle-famille, en particulier Dr \textbf{SOIFFAOUIDDINE Sidi}, le père de ma femme, pour son aide précieuse, tant financière que morale, 
    et pour la confiance qu’il m’a accordée en acceptant de me donner sa fille la plus précieuse.
    % Remerciements
    \newpage
    \begin{center}
        \section*{\textsc{\LARGE Remerciements}}
    \end{center}
    \bigskip
    \textbf{\large Alhamdulillah Rabbi al-‘Alamin.}

    \bigskip

    Ya \textbf{\textsc{\Large Allah}}, merci de m'avoir permis de réaliser ce travail, de m'avoir donné la force et le courage d'affronter toutes les difficultés que j'ai pu rencontrer.
    
    \medskip

    Je tiens à témoigner toute ma reconnaissance et mes remerciements les plus sincères au \textbf{\large Professeur Mountaga LAM}, mon directeur à la DES et encadrant professionnel de ce travail.

    Merci, tout d’abord, de m’avoir permis d’obtenir mon tout premier travail professionnel et salaire de ma vie en m’acceptant à travailler sous votre direction à la DES. Merci pour votre patience, vos encouragements, 
    votre disponibilité, votre confiance et vos nombreux conseils précieux.

    \textit{C’est un immense honneur d’avoir pu travailler avec vous.}

    \medskip

    Je remercie également le \textbf{Professeur Bamba GUEYE}, directeur de l’Office du Bac et encadrant académique de ce travail, pour ce sujet particulièrement stimulant, pour sa disponibilité et ses conseils précieux tout au long de ce travail.

    \medskip

    Je tiens à remercier tout particulièrement \textbf{Dr Souley KANE}, mon directeur de formation, qui m’a permis d’obtenir ce stage et qui a profondément changé ma vie professionnelle en m’ouvrant les portes d’un monde d’opportunités que je n’aurais jamais imaginé intégrer.

    \medskip

    Je n’oublierai pas mes collègues à la DES, en particulier Madame \textbf{MBAYE} et Madame \textbf{PAYE} qui, par leur hospitalité et leur gentillesse, ont été comme des mères pour moi.
    Merci à vous et ainsi qu’à tous mes autres collègues de travail pour votre aide précieuse tout au long de mon stage.

    \medskip

    Je remercie également les membres du jury pour l’attention accordée à ce travail, ainsi que pour leurs remarques pertinentes et enrichissantes.

    \medskip

    Je remercie tous les camarades que j’ai rencontrés au fil de mon parcours pour leur fraternité et leur présence.

    \medskip

    Ma profonde gratitude va à l’endroit de tous les enseignants qui m’ont transmis leur savoir avec passion.Grâce à vous, je suis devenu la personne que je suis aujourd’hui.
    
    \vspace{1cm}
    
    \textit{Merci infiniment à tous, c'était un honneur.}

    % Résumé et Abstract
    \newpage
    \section*{\textsc{Résumé}}
    Ce mémoire présente une analyse approfondie de l'impact des réformes du baccalauréat au Sénégal sur le taux de réussite et l'insertion universitaire. Nous avons utilisé des méthodes statistiques avancées pour évaluer ces effets.
    
    \section*{\textsc{Abstract}}
    This thesis presents a comprehensive analysis of the impact of the baccalaureate reforms in Senegal on the success rate and university integration. We used advanced statistical methods to evaluate these effects.
    
    % Table des matières
    \newpage
    \tableofcontents

    % Liste des codes
    \newpage
    \lstlistoflistings

    % Table des figures 
    \newpage
    \listoffigures

    % Liste des tableaux
    \newpage
    \listoftables

    % Importation des chapitres

    \chapter*{Introduction Générale}
\markboth{\large Introduction Générale}{}
\addcontentsline{toc}{chapter}{Introduction Générale}

\section*{Contexte et justification}
\addcontentsline{toc}{section}{Contexte et justification}

Le système éducatif sénégalais, hérité du modèle français, repose sur une structuration en trois cycles : primaire, moyen et secondaire. 
La fin du secondaire est sanctionnée par le baccalauréat, diplôme pivot marquant l'accès à l'enseignement supérieur et considéré comme le premier grade universitaire. 
Il joue un rôle central, non seulement en tant que certificat de fin d’études, mais aussi comme indicateur de performance du système éducatif.

Malgré les réformes successives visant à adapter ce diplôme aux réalités nationales et aux enjeux contemporains, le baccalauréat au Sénégal reste confronté à un taux de réussite relativement faible, 
oscillant autour de 40\%, loin des standards observés dans des pays comme la France ou le Canada où il s'oscille au tour de 99\% \cite{Mbaye2023}. 
Ce paradoxe entre le nombre croissant de candidats et la stagnation du taux de réussite interroge sur l’efficacité des politiques éducatives mises en œuvre.

Dans ce contexte, plusieurs réformes majeures ont été introduites : l’instauration du baccalauréat arabo-français en 2000, la création du bac arabe en 2013, 
et la transformation de la série G en série STEG (Sciences et Techniques de la Gestion) en 2019. 
Ces réformes traduisent la volonté des autorités d’élargir les opportunités d’accès à l’enseignement supérieur, de diversifier les profils de bacheliers, 
et d'améliorer l'adéquation entre formation et marché du travail. Ce mémoire se propose d’évaluer l’impact réel de ces transformations sur le taux de réussite au baccalauréat et sur le parcours universitaire des bacheliers, notamment à l'UCAD.

\section*{Problématique et Objectifs de l'étude}
\addcontentsline{toc}{section}{Problématique et Objectifs de l'étude}

Les réformes successives du baccalauréat sénégalais, notamment l’introduction du bac arabe en 2013 et de la série STEG en 2019,
ont été motivées par la volonté de moderniser le système éducatif, de mieux adapter l’offre de formation aux besoins socio-économiques du pays et de favoriser l’inclusion. 
Toutefois, leur impact réel reste encore peu documenté de manière rigoureuse.

Une question centrale se pose :

\textbf{Ces réformes ont-elles réellement contribué à améliorer les performances au baccalauréat et à faciliter le parcours universitaire des nouveaux bacheliers ?}

Plusieurs interrogations découlent de cette problématique :
\begin{itemize}
    \item Les réformes ont-elles eu un effet mesurable sur les taux de réussite au baccalauréat, globalement et par série ?
    \item Comment les nouvelles séries, comme le bac arabe ou la série STEG, se positionnent-elles en termes de performances au bac ?
    \item Comment les bacheliers issus des séries arabes évoluent-ils dans l’enseignement supérieur, notamment à l’UCAD ?
    \item Existe-t-il des disparités significatives entre les différentes séries du baccalauréat en termes de réussite et d’insertion à l'UCAD ?
\end{itemize}

À travers une analyse des données du baccalauréat (2006–2024) et des inscriptions à l’UCAD (2002–2024), ce travail poursuit plusieurs objectifs :
\begin{itemize}
    \item \textbf{Analyser l’évolution des taux de réussite au bac}, avec une attention particulière aux périodes post-réformes.
    \item \textbf{Étudier le parcours universitaire des bacheliers (suivi de cohortes)}, en identifiant les départements, filières et performances selon les séries d’origine.
\end{itemize}

Ce travail vise ainsi à combler un manque crucial d’évaluation quantitative des réformes éducatives,
en mobilisant des méthodes statistiques et des visualisations, pour offrir une lecture factuelle et visuelle des effets des réformes, afin d’éclairer la prise de décision dans le secteur éducatif.

\section*{Méthodologie et Organisation du document}
\addcontentsline{toc}{section}{Méthodologie et Organisation du document}

Cette étude repose sur une démarche méthodologique structurée, combinant analyse statistique et visualisation avancée. L’objectif est d’évaluer de manière quantitative et visuelle l’impact des réformes du baccalauréat sur la réussite scolaire et le parcours universitaire.
\newpage
La méthodologie adoptée comprend :

\begin{itemize}
    \item Une \textbf{revue des réformes éducatives} ayant marqué le baccalauréat sénégalais.
    \item L’exploitation de trois grandes bases de données : les \textbf{résultats du baccalauréat} (2006–2024), les \textbf{inscriptions universitaires à l’UCAD} (2002–2024) et les \textbf{résultats universitaires à l’UCAD} (2011–2024).
    \item Une \textbf{analyse Exploratoire}, appuyée par des représentations visuelles riches afin de faciliter la compréhension et l’interprétation des résultats.
    \item Un \textbf{suivi longitudinal des cohortes} de bacheliers à l’université, avec une attention particulière portée aux différences selon les séries d’origine.
    \item Une \textbf{restitution interactive} à l’aide de tableaux de bord dynamiques réalisés avec Power BI.
\end{itemize}


Le document est organisé en six chapitres principaux, suivis d'une section dédiée à la conclusion et aux recommandations, suivant la logique de cette démarche :

\begin{enumerate}
    \item \textbf{État de l’art} : Présentation des réformes du baccalauréat et revue des études antérieures sur le sujet.
    \item \textbf{Données et outils utilisés} : Description des sources de données et des outils mobilisés.
    \item \textbf{Nettoyage et préparation des données} : Traitement, fusion et structuration des données pour les rendre exploitables.
    \item \textbf{Analyse du taux de réussite au bac} : Analyse visuelle du taux de réussite au baccalauréat à travers des graphiques.
    \item \textbf{Analyse du parcours universitaire et suivi de cohortes} : visualisation du cheminement des bacheliers à l’UCAD.
    \item \textbf{Restitution interactive et visualisation} : Présentation des dashboard Power BI, incluant les filtres dynamiques, mesures, colonnes conditionnelles et visualisations interactives.
\end{enumerate}

\textbf{Conclusion et recommandations} : Bilan général de l’étude, limites rencontrées et propositions pour renforcer l’efficacité des politiques éducatives. % Importation de l'introduction générale

    \chapter{État de l’art}

\section{Introduction}

Depuis sa création, comme énoncé pécedemment, le baccalauréat a connu de nombreuses évolutions, tant dans sa structure que dans sa finalité.
Ce chapitre propose une revue des principales réformes ayant marqué le système du baccalauréat au Sénégal, en mettant en lumière leur contexte, leur portée et leurs objectifs.
Il vise également à situer l'organisation de notre étude dans ce cadre évolutif. 

\section{Historique et rôle du baccalauréat}

Le baccalauréat a été inventé en France au XIXe siécle par un décret de l'empereur Napoléon Ier en 1808.
Au commencement le baccalauréat s'appelait le \textbf{Bachot}.
Son étymologie \textit{ "bacca laurea"} désigne en latin médiéval, \textbf{la couronne de laurier} remise aux vainqueurs \cite{bacHistorique}.

Dans le contexte sénégalais, il a été progressivement intégré au système éducatif colonial, puis nationalisé après l’indépendance. 
Il sanctionne la fin du cycle secondaire et donne accès à l’université, devenant ainsi un indicateur majeur de performance du système éducatif.

\section{Organisation du baccalauréat : structure et acteurs}

Le baccalauréat Sénégalais est organisé par \textbf{l'Office du baccalauréat}, rattaché à l'Université Cheikh Anta Diop de Dakar (UCAD), ce qui renforce l'idée qu'il constitue le premier diplôme universitaire.
Il mobilise chaque année plusieurs milliers d’enseignants, surveillants, correcteurs et examinateurs.

Il se décline en différentes séries réparties en trois filières distinctes : \textbf{littéraire}, \textbf{scientifique} et \textbf{tertiaire}, afin de répondre à la diversité des profils et des parcours de formation. 

L’examen repose sur des programmes définis par le ministère en charge de l’éducation nationale, avec une centralisation de l’organisation et des corrections.

\section{Réformes du baccalauréat au Sénégal}

Depuis les années 1990, plusieurs réformes ont été introduites pour adapter le baccalauréat aux réalités sociolinguistiques, 
économiques et pédagogiques du pays.

\subsection{Réformes de 1995 (décret n° 95-947)}

Certainement la réforme la plus marquante, elle s'inscrit dans le cadre de la concertation nationale sur l'enseignement supérieur tenu le 9 décembre 1995, 
dont sont issues les propositions de la commission qui avait été chargée de réfléchir sur l'examen du baccalauréat \cite{decret95}.

Elle a introduit les changements majeurs dans la structure des séries telles qu'on les connait aujourd'hui, 
notamment par l'article 7 du décret établissant les choix des séries que les candidats devront choisir au moment de leurs inscriptions :

\begin{itemize}
    \item Série L1 : Langues et Civilisations
    \item Série L2 : Sciences sociales et humaines
    \item Série G : Techniques quantitatives d'économie et de gestion
    \item Série S1 : Sciences exactes (Mathématiques et Physique)
    \item Série S2 : Sciences expérimentales
    \item Série S3 : Sciences et Techniques
    \item Série T1 : Fabrication mécanique
    \item Série T2 : Electrotechnique-Electronique
\end{itemize}

C'est pour modifier et compléter cette article 7 du décret n°1995-947 que toutes les réformes étudiées dans ce mémoire ont été introduites.

\subsection{Réformes de 2000 (décret n° 2000-585)}

Apparue très tôt dans l'élémentaire, la langue arabe sera reconnue comme langue vivante étrangère au Sénégal, laissée au choix de l'élève dans l'enseignement moyen, puis secondaire.

Parallélement à cette évolution dans l'enseignement public, l'initiative privée, portée par une demande socio-culturelle, a donné naissance à un système d'enseignement arabe encore embryonnaire.
Ainsi, avec la multpilication des établissements exclusivement arabe ou bilingue franco-arabe, la nécessité d'encadrer ce phénomène a conduit l'État sénégalais à mettre en place un référentiel de diplôme : 
c'est la création du \textbf{Certificat arabe} et du \textbf{BFEM arabe} \cite{decret2000}.

La création du collège public franco-arabe \textbf{Mouhamadou Fadilou Mbacké} de Dakar en 1963 procéde de cette volonté de répondre à cette demande sociale. Un second cycle y a été mis en place, offerant les même perspectives que celles proposées aux autres collègiens.
Ainsi, le décret n°2000-585 du 6 juillet 2000 a introduit le baccalauréat option arabe qui modifie l'article 7 du décret n° 95-947 en un article 7 bis en intégrant les baccalauréats option arabe suivants : 

\begin{itemize}
    \item Langues et Sciences sociales : LA
    \item Sciences fondamentales : S1A
    \item Sciences appliquées : S2A
\end{itemize}

\subsection{Réformes de 2013 (décret n° 2013-913)}

\textit{"Le baccalauréat organisé (matérialisé) par le décret n°2000-586 du 20 juillet 2000, 
n'a pas atteint tous ses objectifs car des écoles privées franco-arabes et des instituts islamiques se multiplient entraînant un manque de contrôle
sur les programmes enseignés aini qu'une prolifération de diplômes qui empêche l'existence d'un standard commun à toutes les études secondaires du pays"}\cite{decret2013}.

C'est pour encadrer ce phénomène que, tout comme les études secondaires franco-arabes, les études secondaires arabes seront sanctionnées par les baccalauréats arabes. 
Ces derniers sont définis dans un article 7 ter, qui complète l'article 7 bis du décret n° 2000-585, en intégrant les séries suivantes :

\begin{itemize}
    \item Littératures et Civilisations arabes : L-AR 
    \item Mathématiques et Sciences Physique : S1-AR
    \item Sciences expérimentales : S2-AR
\end{itemize} 

\subsection{Réformes de 2019(décret n° 2019-645)}

À la difference des précédentes réformes, celle-ci à pour but de renforcer la série G aux exigences des Formations Professionnelles et Techniques(FPT)
et des Instituts Supérieurs d'Enseignement Professionnel (ISEP).

\textit{"C'est dans ce cadre qu'il est proposé que la série G (éconmie et gestion) soit transformée en série technologique (sciences et technologique de l'économie et de la gestion : STEG)
visant essentiellement à installer chez les élèves les compétences en associant la culture générale et la technologie"}\cite{decret2019}.

\section{Travaux antérieures }

\section{Approche méthodologique et innovante de l'étude}

Ce travail s’inscrit dans le cadre d’un stage à la Direction des Études et des Statistiques (DES) de l’UCAD, en partenariat avec l’Office du Baccalauréat. 
Il combine une analyse data-driven et des méthodes de machine learning pour évaluer l’impact des réformes sur deux axes :
\begin{enumerate}
    \item \textbf{performance académique}: Analyse de l'évolution du taux de réussite au bac de 2006 à 2024 globale et des série concerner, via des modèles de régression.
    \item \textbf{performance universitaire}: Une suivit des cohortes ici des séries concerner du bac 
\end{enumerate}
Un \textbf{tableau de bord interactif} sera mis en place pour visualiser les résultats de l'analyse et permettre une exploration dynamique des données.

\section{Conclusion}

Ce premier chapitre a permis de retracer l’évolution historique et structurelle du baccalauréat au Sénégal, en mettant en lumière les principales réformes qui l’ont façonné depuis les années 1990. 
Ces réformes, souvent motivées par des enjeux pédagogiques, socioculturels ou économiques, ont profondément modifié les séries existantes, introduit de nouvelles filières, et adapté l’examen aux réalités locales. 
Elles traduisent la volonté constante des autorités de moderniser et de diversifier le système éducatif pour mieux répondre aux besoins de la société sénégalaise. Cette contextualisation est essentielle pour comprendre la portée de notre étude, 
qui se propose d’évaluer l’impact de ces réformes à travers une approche analytique et prédictive, fondée sur l’exploitation des données réelle. % Importation du chapitre 1

    \chapter{Données et outils utilisés}

\section{Introduction}

Ce chapitre présente l'approche méthodologique adoptée dans le cadre de ce projet.
Nous y détaillons les différentes sources de données exploitées, les étapes de  Nettoyage et de préparation des données, 
ainsi que les outils utilisés pour leur traitement, leur analyse et leur visualisation. 

Comprendre la nature des données est essentiel pour interpréter correctement les résultats obtenus. 
Par ailleurs, le choix des outils d’analyse joue un rôle déterminant dans la rigueur et la reproductibilité des analyses menées.

\section{Sources et Description des données}
\subsection{Sources de données}
\hspace{-1cm}
Ce projet s’appuie sur trois principales sources de données fournies par des institutions officielles : 
\begin{itemize}
    \item \textbf{l’Office du Baccalauréat du Sénégal};
    \item \textbf{la Direction de l’Informatique et des Systèmes d’Information (DISI)} de l’UCAD;
\end{itemize}

Ces sources couvrent l’ensemble du parcours des étudiants, depuis leur réussite au bac jusqu’à leur évolution dans l’enseignement supérieur.

\subsection{Description des données}

\subsubsection{Résultats du Baccalauréat (2006–2024)}

Ces données ont été fournies par l’Office du Baccalauréat du Sénégal. 
Elles contiennent les informations relatives aux candidats (année, série, résultat, mention, session, etc.). 
Elles sont utiles pour suivre l’évolution du taux de réussite au bac et évaluer l’effet des réformes introduites dans certaines séries.

\subsubsection{Inscriptions universitaires à l’UCAD (2002–2024)}

Ces données proviennent de la Direction de l’Informatique et des Systèmes d’Information (DISI) de l’UCAD. 
Elles incluent les informations sur les étudiants inscrits : année d’inscription, série d’origine, sexe, âge, établissement, etc. 
Elles permettent d’analyser les flux d’entrée à l’université selon le profil des bacheliers.

\subsubsection{Résultats académiques à l’UCAD (2010–2023)}

Également issues de la DISI de l’UCAD, ces données donnent des détails sur la performance académique des étudiants :
moyenne annuelle, crédits validés, session, mention et résultat (passage, redoublement). 
Elles servent à évaluer la réussite universitaire en fonction des parcours scolaires initiaux.

\newpage
\section{Outils et Technologies utilisés}

\subsection{Python et ses bibliothèques }

\includegraphics[width=3cm]{images/python.png}

Python est un langage de programmation open source, interprété, 
simple à apprendre et largement utilisé dans le domaine scientifique et technique.
Python s’est imposé comme l’un des outils les plus puissants pour l’analyse de données et le développement de modèles d’intelligence artificielle, 
notamment en raison de la richesse de ses bibliothèques spécialisées \cite{python}.

Dans le cadre de cette étude, Python a été utilisé à la fois pour le traitement, l’exploration et la visualisation des données,
ainsi que pour le développement de modèles de machine learning.

\subsubsection{bibliothèques utilisées}

\includegraphics[width=3cm]{images/Pandas_logo.png}

Pandas est une bibliothèque Python spécialisée dans la manipulation et l’analyse de données. 
Elle offre des structures de données flexibles et efficaces, notamment les \textbf{DataFrame}, 
qui permettent de gérer facilement des tableaux de données similaires à ceux d'Excel ou d'une base de données\cite{pandas}.

C’est sans doute le package le plus utilisé en science des données. Grâce à ses nombreuses fonctionnalités,
il ma permet de charger, filtrer, nettoyer et transformer les jeux de données de manière rapide et intuitive.

\includegraphics[width=4cm]{images/Matplotlib.png}

Matplotlib est une bibliothèque Python dédiée à la visualisation de données. 
Elle permet de créer une grande variété de graphiques statiques. 
Son interface simple et sa compatibilité avec les structures de données 
comme les DataFrame de pandas en font un outil de choix pour représenter visuellement les résultats d’une analyse \cite{matplotlib}.

\includegraphics[width=3cm]{images/sckitlearn.png}

Scikit-learn est une bibliothèque Python dédiée au machine learning. 
Elle offre des outils simples et efficaces pour appliquer des modèles de classification, de régression et de clustering \cite{scikitLearn}.

Dans mon projet, je l’ai utilisée pour prédire le taux de réussite au baccalauréat. 
Elle m’a aussi permis d’évaluer les performances des modèles grâce à des métriques comme (RMSE).

\includegraphics[width=5cm]{images/statsmodes.png}

Statsmodels est une bibliothèque Python conçue pour l’estimation de modèles statistiques. 
Elle est particulièrement utilisée pour les analyses de régression, les séries temporelles et les tests statistiques \cite{statsmodels}.

Dans mon projet, Statsmodels m’a servi à réaliser des régressions linéaires et à effectuer des tests d’hypothèses. 
Elle m’a permis d’interpréter les relations entre les variables, grâce à des résultats détaillés incluant les coefficients, 
les p-values et les intervalles de confiance.

\subsection{Power BI}

\includegraphics[width=4cm]{images/powerbi.png}

Power BI est un outil de visualisation et d’analyse de données développé par Microsoft. 
Il permet de créer des tableaux de bord interactifs et dynamiques à partir de diverses sources de données, 
facilitant ainsi l'exploration visuelle et la prise de décision basée sur les données \cite{powerBI}.

Dans le cadre de mon projet, j’ai utilisé Power BI pour concevoir un tableau de bord interactif regroupant les statistiques du baccalauréat de 2006 à 2024. 
Ce tableau de bord, présenté dans le chapitre 6, est destiné à l’Office du Baccalauréat. Il a pour objectif de faciliter l’analyse des données historiques du bac et de servir de support à toute nouvelle étude portant sur l’évolution du système éducatif.


\section{Conclusion}

Ce chapitre a permis de présenter les données exploitées ainsi que les outils mobilisés pour leur traitement, leur analyse et leur visualisation. 
Les trois jeux de données, provenant de sources officielles telles que l’Office du Baccalauréat et la DISI de l’UCAD, 
offrent une base solide pour évaluer l’impact des réformes éducatives sur les résultats au bac et l’insertion universitaire. 
Par ailleurs, l’utilisation d’outils puissants comme Python et ses bibliothèques spécialisées, ainsi que Power BI pour la visualisation, 
garantit une analyse rigoureuse, reproductible et accessible. Ces ressources forment ainsi le socle méthodologique sur lequel reposent les analyses menées dans les chapitres suivants. % Importation du chapitre 2

    \chapter{Nettoyage et préparation des données}

\section{Introduction}
Avant d’entamer toute analyse statistique, une étape cruciale consiste à examiner, nettoyer et structurer les données. 
Bien que les jeux de données utilisés dans ce projet proviennent de sources officielles telles que l’Office du Bac et la DISI/UCAD, et présentent globalement une structure cohérente, un travail de nettoyage s’est avéré nécessaire.

Contrairement à l’hypothèse initiale selon laquelle les données ne nécessiteraient qu’un traitement minimal, plusieurs opérations classiques de nettoyage ont finalement été réalisées. 
Cela inclut la détection et le traitement de valeurs aberrantes, la gestion des doublons, ainsi que le traitement ciblé des valeurs manquantes. 
  Par exemple, dans les résultats universitaires, certaines valeurs nulles traduisent l'absence à une évaluation, mais d'autres relevaient d’incohérences ou de saisies incomplètes qu’il a fallu corriger ou exclure selon le contexte.

En parallèle, des opérations de préparation ont été menées pour structurer les données en vue de l’analyse. 
Celles-ci comprennent la fusion de bases de données complémentaires (Les Inscrits et Résultats de UCAD), la standardisation des formats de certaines variables, ainsi que la création de nouvelles colonnes dérivées utiles à l’étude.

Les sous-sections suivantes détaillent les étapes spécifiques effectuées lors de ce processus de nettoyage et de préparation.


\newpage
\section{Préparation de données des résultats du baccalauréat}

Les données relatives aux résultats du baccalauréat ont été centralisées dans une table unique contenant plus de deux millions d’enregistrements, couvrant la période de 2006 à 2024. 
Le code ci-dessous (~\ref{lst:info_bac}) donne un aperçu général de la structure du DataFrame après concaténation des fichiers annuels.
\begin{lstlisting}[language=Python,
    caption=Informations général du DataFrame,
    label=lst:info_bac,
    basicstyle=\ttfamily\small,
    backgroundcolor=\color{gray!10}
]
<class 'pandas.core.frame.DataFrame'>
RangeIndex: 2236490 entries, 0 to 2236489
Data columns (total 15 columns):
 #   Column           Dtype 
---  ------           ----- 
 0   nom              object
 1   prenom           object
 2   numero_table     object
 3   serie            object
 4   sexe             object
 5   age              object
 6   etablissement    object
 7   type_candidat    object
 8   resultat         object
 9   acad_provenance  object
 10  moy_finale       object
 11  mention          object
 12  abs              object
 13  exclusion        object
 14  year             object
dtypes: object(15)
memory usage: 255.9+ MB
\end{lstlisting}

On y retrouve des informations telles que le prénom, le nom, le numéro de table, la série, le sexe, l’âge, l'établissement, les résultats, la moyenne finale, la mention, etc. 
On observe un total de 14 colonnes et 2236490 lignes.
\newpage
\subsection{Doublons dans la clé primaire \texttt{numero\_table}}

\begin{lstlisting}[language=Python,
    caption=Nombre de numero\_table en doublon par année,
    label=lst:doublons,
    basicstyle=\ttfamily\small,
    backgroundcolor=\color{gray!10}
]
Annee 2006 : 0 numeros apparaissent au moins deux fois
Annee 2007 : 0 numeros apparaissent au moins deux fois
Annee 2008 : 371 numeros apparaissent au moins deux fois
Annee 2009 : 0 numeros apparaissent au moins deux fois
Annee 2010 : 0 numeros apparaissent au moins deux fois
Annee 2011 : 1 numeros apparaissent au moins deux fois
Annee 2012 : 2 numeros apparaissent au moins deux fois
Annee 2013 : 0 numeros apparaissent au moins deux fois
Annee 2014 : 0 numeros apparaissent au moins deux fois
Annee 2015 : 0 numeros apparaissent au moins deux fois
Annee 2016 : 0 numeros apparaissent au moins deux fois
Annee 2017 : 0 numeros apparaissent au moins deux fois
Annee 2018 : 0 numeros apparaissent au moins deux fois
Annee 2019 : 0 numeros apparaissent au moins deux fois
Annee 2020 : 0 numeros apparaissent au moins deux fois
Annee 2021 : 0 numeros apparaissent au moins deux fois
Annee 2022 : 0 numeros apparaissent au moins deux fois
Annee 2023 : 0 numeros apparaissent au moins deux fois
Annee 2024 : 0 numeros apparaissent au moins deux fois 
\end{lstlisting}

Les résultats révèlent que la majorité des années ne présentent aucun doublon. Toutefois, quelques
années, comme 2008 (371 doublons), 2011 (1) et 2012 (2) comportent des cas où deux
candidats partagent le même numéro de table dans la même année. Cela peut s’expliquer par des
erreurs de saisie ou des anomalies administratives.

\begin{table}[h]
\scriptsize
\centering
\caption{lignes avec les numéros en doublon en 2012}
\label{tab:doublon}
\begin{tabular}{llllllr}
\toprule
nom & numero\_table & serie & sexe & age & etablissement & moy\_finale \\
\midrule
CISSE & 56234 & L'1 & M & 17 & LYCEE CHARLES DEGAULLE & 11,00 \\
KANE & 56234 & L2 & M & 21 & LYCEE DE ROSS - BETHIO & 07,07 \\
BOYE & 56513 & L'1 & F & 19 & LYCEE EL HADJ OMAR FOUTIYOU TALL & 05,58 \\
CAMARA & 56513 & L'1 & M & 23 & STRUCTURE D'ENTRE AIDE DU LYCEE EL HADJ OMAR TALL & 07,17 \\
\bottomrule
\end{tabular}
\end{table}

Le tableau (table~\ref{lst:doublon}) illustre un exemple concret pour l’année 2012, où deux élèves de centres différents ont le même numéro de table mais des informations distinctes. 
Ces cas ont été traités avec prudence dans la suite des analyses.

\textbf{Remarque :}

On remarque qu’il ne s’agit pas de doublons exacts, mais de personnes différentes à qui le même numéro de table a été attribué.

\newpage
\subsection{Les valeurs manquantes dans les colonnes}

\begin{table}[h]
\hspace{5cm}
\caption{Valeurs manquantes dans les colonnes du DataFrame des résultats du bac}
\begin{tabular}{lr}
\toprule
Colonnes & nb\_valeur\_null \\
\midrule
nom & 1 \\
prenom & 1 \\
numero\_table & 3 \\
serie & 5442 \\
sexe & 1 \\
age & 1 \\
etablissement & 1 \\
type\_candidat & 1 \\
resultat & 3977 \\
acad\_provenance & 1 \\
moy\_finale & 11 \\
mention & 945542 \\
abs & 1 \\
exclusion & 1 \\
year & 0 \\
\bottomrule
\end{tabular}
\end{table}

\subsubsection{Valeurs manquantes dans la colonne \texttt{moy\_finale}}

\begin{table}[h]
\hspace{5cm}
\caption{Valeurs manquantes dans la colonne moy\_finale}
\begin{tabular}{lrllr}
\toprule
nom & numero\_table & resultat & moy\_finale \\
\midrule
NaN & 42379 & NaN & NaN \\
XXXXXXXX & 41803 & NaN & NaN \\
XXXXXXXX & 42043 & NaN & NaN \\
XXXXXXXX & 27911 & NaN & NaN \\
XXXXXXXX & 24635 & NaN & NaN \\
XXXXXXXX & 17167 & NaN & NaN \\
XXXXXXXX & 17669 & NaN & NaN \\
XXXXXXXX & 1054 & NaN & NaN \\
NDIAYE & NaN & NaN & NaN \\
FAYE & NaN & NaN & NaN \\
KANE & NaN & NaN & NaN \\
\bottomrule
\end{tabular}
\end{table}

\subsubsection{Valeurs manquantes dans la colonne \texttt{resultat}}

\textbf{Les valeurs possibles dans la colonne \texttt{resultat}}

\newpage
\subsection{Correction du type de la colonne \texttt{moy\_finale}}

Comme on peut le constater dans le code présenté (listing~\ref{lst:info_bac}), le type initial de la colonne \texttt{moy\_finale} n'était pas exploitable tel quel. 
En effet, certaines valeurs contenaient des tirets, d'autres utilisaient des virgules comme séparateur décimal, et certaines étaient tout simplement vides ou non numériques.

\begin{lstlisting}[language=Python,
    caption=Correction du type de la colonne moy\_finale,
    label=lst:moy_finale_type,
    basicstyle=\ttfamily\small,
    backgroundcolor=\color{gray!10}
]
all_data_filtre['moy_finale'] = (
    all_data_filtre['moy_finale']
    .astype(str) # Convertir en chaine de caracteres
    .str.replace('-', '', regex=False) # Supprimer les tirets
    .str.replace(',', '.', regex=False) # les virgules en points
    .str.strip() # Supprimer les espaces au debut et fin de chaine
    .replace({'': None, 'nan': None}) # Supprimer les chaines vides
    .astype(float) # Convertir en float
)
\end{lstlisting}

Pour rendre cette colonne exploitable statistiquement, plusieurs opérations de nettoyage ont été effectuées :
\begin{itemize}
\item Suppression des tirets \texttt{'-'}.
\item Remplacement des virgules \texttt{','} par des points \texttt{'.'}.
\item Suppression des espaces superflus.
\item Conversion des chaînes vides ou non valides (\texttt{''}, \texttt{'nan'}) en valeurs manquantes.
\item Conversion finale de la colonne en type \texttt{float}.
\end{itemize}

Ces étapes garantissent la cohérence de la variable \texttt{moy\_finale} pour les analyses statistiques à venir.

\newpage
\subsection{Création des colonnes \texttt{admis} et \texttt{session}}

Pour faciliter les analyses, deux nouvelles colonnes ont été dérivées de la variable \texttt{resultat} :
\begin{itemize}
\item La colonne \texttt{admis}, indiquant si un candidat est admis ou non. Les codes \texttt{'111'} et \texttt{'101'} ont été interprétés comme signifiant « admis ».
\item La colonne \texttt{session}, précisant s’il s’agit du premier ou du second tour, en se basant sur les mêmes codes.
\end{itemize}

\begin{lstlisting}[language=Python,
    caption=Création de nouvelles colonnes,
    label=lst:creation_colonnes,
    basicstyle=\ttfamily\small,
    backgroundcolor=\color{gray!10}
]
# Creation de la colonne 'admis' de 'resultat'
all_data['admis'] = all_data['resultat'].apply(lambda x: 'admis' 
                            if pd.notna(x) and x in ['111', '101'] 
                            else 'non admis')

# Creation de la colonne 'session' de 'admis'
all_data['session'] = all_data['resultat'].apply(lambda x: '1er Tour' 
                            if pd.notna(x) and x == '111' 
                            else ('2e Tour' if pd.notna(x) and x == '101' 
                                    else ''))
\end{lstlisting}

Ce pré-traitement permet d’améliorer la lisibilité des résultats, notamment dans les agrégations et visualisations.

\newpage
\section{Fusion des données des Inscrits et des Résultats de l’UCAD}

\subsection{Données des inscriptions à l’UCAD}

Les données d’inscription utilisées dans cette étude couvrent la période de 2002 à 2024. 
Cependant, pour garantir la cohérence avec les données de résultats (disponibles uniquement de 2011 à 2024), 
nous avons retenu uniquement les inscriptions allant de 2011 à 2023.

La sortie de code \ref{lst:inscription_ucad} présente un aperçu global de la base de données des inscriptions à l’UCAD pour les années universitaires allant de 2011 à 2023. 
Elle contient 1 141 120 enregistrements répartis sur 20 variables. Les variables \textit{NUMERO} et \textit{ANNEE UNIVERSITAIRE}, 
constituent les clés essentielles pour la fusion avec la base des résultats académiques.


\begin{lstlisting}[language=Python,
    caption=Info global du data des inscriptions, 
    label=lst:inscription_ucad, 
    basicstyle=\ttfamily\footnotesize, 
    backgroundcolor=\color{gray!10}
]
<class 'pandas.core.frame.DataFrame'>
RangeIndex: 1141120 entries, 0 to 1141119
Data columns (total 14 columns):
 #   Column                 Non-Null Count    Dtype  
---  ------                 --------------    -----  
 0   NUMERO                 1141120 non-null  object 
 1   SEXE                   1141120 non-null  object 
 2   ANNEE_BACC             1136174 non-null  float64
 3   NATIONALITE            1141120 non-null  object 
 4   SERIE_BACC             1109114 non-null  object 
 5   ETABLISSMENT_CODE      1141120 non-null  object 
 6   NIVEAU_SECTION         1141120 non-null  object 
 7   ANNEE_INSCRIPTION      1141120 non-null  int64  
 8   ANNEE_UNIVERSITAIRE    1141120 non-null  object 
 9   TYPE_FORMATION         1141120 non-null  object 
 10  CODE_NIVEAU            1141120 non-null  int64  
 11  NIVEAU LMD ET NON LMD  1141120 non-null  object 
 12  SYSTEME                1141120 non-null  object 
 13  DEPARTEMENT FORMATION  1141120 non-null  object 
dtypes: float64(1), int64(2), object(11)
memory usage: 121.9+ MB
\end{lstlisting}

\newpage
\subsection{Données des résultats de l’UCAD}

La sortie de code \ref{lst:resultat} présente les informations générales de la base de données contenant les résultats universitaires des étudiants de l’UCAD. 
Cette base couvre la période allant de 2010 à 2024, et concerne uniquement les établissements ayant effectué leurs délibérations sur la plateforme institutionnelle de la DISI. 
On y retrouve un total de 753 828 enregistrements répartis sur 27 colonnes.

\begin{lstlisting}[language=Python,
    caption=Info global du data des résultats, 
    label=lst:resultat_ucad, 
    basicstyle=\ttfamily\footnotesize, 
    backgroundcolor=\color{gray!10}
]
<class 'pandas.core.frame.DataFrame'>
RangeIndex: 753693 entries, 0 to 753692
Data columns (total 3 columns):
 #   Column               Non-Null Count   Dtype 
---  ------               --------------   ----- 
 0   NUMERO               753693 non-null  object
 1   ANNEE UNIVERSITAIRE  753693 non-null  object
 2   RESULTAT             753562 non-null  object
dtypes: object(3)
memory usage: 17.3+ MB
\end{lstlisting}

\newpage
\subsection{Fusion des données d’inscription et de résultats}

Pour fusionner les bases de données d’inscription et de résultats, 
les colonnes \textit{NUMERO} et \textit{ANNEE UNIVERSITAIRE} ont été utilisées comme clés de jointure. 

\begin{lstlisting}[language=Python,
    caption=Jointure des données d’inscription et de résultats,
    label=lst:jointure_ucad,
    basicstyle=\ttfamily\small,
    backgroundcolor=\color{gray!10}
]
df_final = pd.merge(df_inscrit, # Premier DataFrame
                df_resultat, # Deuxieme DataFrame
                on=['NUMERO', 'ANNEE UNIVERSITAIRE'], # Cles de jointure
                how='left' # garde toutes les lignes de df_inscrit
                ) 
\end{lstlisting}

La fusion a été réalisée à l’aide de l’option \texttt{how='left'} pour conserver l’ensemble des inscrits, 
y compris ceux pour lesquels aucun résultat académique n’est disponible.

\begin{lstlisting}[language=Python,
    caption=Info global du data des inscriptions et résultats, 
    label=lst:inscription_resultat_ucad, 
    basicstyle=\ttfamily\footnotesize, 
    backgroundcolor=\color{gray!10}
]
<class 'pandas.core.frame.DataFrame'>
RangeIndex: 1215837 entries, 0 to 1215836
Data columns (total 15 columns):
 #   Column                 Non-Null Count    Dtype  
---  ------                 --------------    -----  
 0   NUMERO                 1215837 non-null  object 
 1   SEXE                   1215837 non-null  object 
 2   ANNEE_BACC             1210891 non-null  float64
 3   NATIONALITE            1215837 non-null  object 
 4   SERIE_BACC             1183525 non-null  object 
 5   ETABLISSMENT_CODE      1215837 non-null  object 
 6   NIVEAU_SECTION         1215837 non-null  object 
 7   ANNEE_INSCRIPTION      1215837 non-null  int64  
 8   ANNEE UNIVERSITAIRE    1215837 non-null  object 
 9   TYPE_FORMATION         1215837 non-null  object 
 10  CODE_NIVEAU            1215837 non-null  int64  
 11  NIVEAU LMD ET NON LMD  1215837 non-null  object 
 12  SYSTEME                1215837 non-null  object 
 13  DEPARTEMENT FORMATION  1215837 non-null  object 
 14  RESULTAT               750198 non-null   object 
dtypes: float64(1), int64(2), object(12)
memory usage: 139.1+ MB
\end{lstlisting}

Cela explique la présence d’un grand nombre de valeurs manquantes dans la colonne \textit{RESULTAT}, 
notamment pour les étudiants dont les résultats n'ont pas encore été délibérés ou publiés sur la plateforme.

\newpage

\section{Filtage des données}
\section{Conclusion} % Importation du chapitre 3

    \chapter{Analyse du taux de réussite au bac}

\section{Introduction}

Ce chapitre est consacré à l’analyse du taux de réussite au baccalauréat au Sénégal sur la période 2006–2024. 
L’objectif est de mettre en lumière les tendances globales et spécifiques aux séries concernées par les réformes, à travers une exploration visuelle des données.

Nous débuterons par l’évolution du nombre d’inscrits et du taux de réussite global, avant de nous concentrer sur les séries spécifiques analysées dans ce travail, notamment les séries arabes, franco-arabes, STEG, G. 

Des visualisations comparatives permettront d’identifier les ruptures, dynamiques et écarts entre les différentes filières. 
En complément, une modélisation prédictive sera réalisée afin d’estimer l’évolution théorique du taux de réussite en l’absence des réformes, offrant ainsi une base de comparaison pour mesurer leur impact.

\section{Analyse de globale}

\newpage
\subsection{Évolution des effectifs}

\begin{figure}[ht]
\centering
\caption{Évolution du nombre d'inscrits, présents et admis au bac (2006-2024)}
\includegraphics[width=1\textwidth]{figure/Inscrits_bac.png}
\label{fig:inscrits_admis}
\end{figure}

La figure (Figure~\ref{fig:inscrits_admis}) représente l’évolution du nombre d’inscrits, de présents et d’admis au baccalauréat entre 2006 et 2024. 
L’observation met en évidence deux phases temporelles distinctes.

\textbf{Phase 1 — Croissance rapide (2006–2015)}  

Durant cette période, le système éducatif sénégalais connaît une expansion notable du nombre de candidats au bac. 
Le nombre d’inscrits passe de 42 095 en 2006 à 147 711 en 2015, soit une croissance d’environ \textbf{250\,\%} en 9 ans.

Les admis augmentent également mais de manière moins rapide, passant de 20 411 en 2006 à 45 356 en 2015, soit une hausse d’environ \textbf{125\,\%}. 
Ce décalage entre l’augmentation des présents et celle des admis révèle un déséquilibre croissant sur les performances au baccalauréat.

\textbf{Phase 2 — Stabilisation relative (2015–2024)}  

À partir de 2015, le nombre d’inscrits se stabilise autour de 150 000 à 160 000 candidats par an. 
La croissance ralentit, marquant une saturation ou une stabilisation des flux en fin de cycle secondaire. 
Une chute brutale du nombre de présents est toutefois observée en 2021, ce qui peut être associé aux perturbations causées par la pandémie de COVID-19.

Les admissions au baccalauréat continuent d'afficher une croissance modérée, atteignant 78 246 en 2024. Toutefois, cette évolution reste notablement faible comparée au nombre de candidats inscrits et présents.

\subsection{Évolution du taux de réussite}

\begin{figure}[ht]
\centering
\caption{Évolution du taux de réussite au bac (2006-2024)}
\includegraphics[width=1\textwidth]{figure/taux_bac.png}
\label{fig:taux_reussite}
\end{figure}

La figure (Figure~\ref{fig:taux_reussite}) illustre l’évolution du taux de réussite au baccalauréat sénégalais entre 2006 et 2024. 
L’analyse révèle des fluctuations marquées, avec des périodes de déclin et de reprise.


\textbf{Période 2006–2017 — Baisse significative}

Durant cette phase, le taux de réussite connaît une chute alarmante, passant de 50\% en 2006 à 31\% en 2017. 
Cette diminution de près de 20 points en onze ans souligne des difficultés structurelles.

\textbf{Période 2017–2024 — Progression notable}

À partir de 2017, le taux de réussite montre une nette amélioration, atteignant 50\% en 202, avec un pic record 57\% en 2021(ce taux élevé peut s'expliquer par la chute brutale du nombre de présents liée à la pandémie, puisqu'il est calculé en fonction des candidats ayant effectivement passé l’examen). 
Toutefois, une légère baisse est observée entre 2021 et 2024, ce qui peut être source d’inquiétude.

\bigskip

Malgré une légère amélioration récente, le taux de réussite au baccalauréat demeure généralement trop faible pour un système éducatif aspirant à la performance.

\newpage
\section{Analyse de la transition entre les séries G et STEG}

La réforme du baccalauréat technique au Sénégal, concrétisée par le décret de 2019, marque un tournant important dans l’organisation de la série Techniques quantitatives d’économie et de gestion.
Elle prévoit \textbf{la suppression progressive de la série G}, au profit de la nouvelle série \textbf{STEG}.
Cette dernière vise à mieux articuler les enseignements généraux et technologiques afin de développer chez les élèves des compétences applicables dans l’enseignement supérieur et la vie professionnelle.

Conformément au décret, la série G a continué à être organisée jusqu’en 2022, date de sa dernière session. À partir de 2023, seule la série STEG est maintenue. 
Cette transition permet d’évaluer les effets de la réforme en comparant les performances (nombre d’inscrits et taux de réussite) de la série STEG à celles de son prédécesseur, la série G.

\subsection{Évolution du nombre d'inscrits dans les séries G et STEG}

\begin{figure}[ht]
\centering
\caption{Évolution du nombre d'inscrits dans les séries G et STEG (2006-2024)}
\includegraphics[width=1\textwidth]{figure/Inscrits_bac_STEG.png}
\label{fig:inscrits_STEG}
\end{figure}

La figure (Figure~\ref{fig:inscrits_STEG}) montre l’évolution du nombre d’inscrits au baccalauréat dans les séries G et STEG entre 2006 et 2024. 

Jusqu’en 2018, seule la série G est présente, avec un pic d’inscription atteint entre 2012 et 2014. À partir de 2019, la série STEG est introduite et connaît une croissance rapide, tandis que les effectifs de la série G diminuent progressivement, jusqu’à leur extinction en 2023.

Ce croisement entre les deux courbes montre clairement une transition bien gérée sur le plan quantitatif, avec un transfert progressif des effectifs. 
Dès 2020, les inscriptions en STEG dépassent celles de la série G, traduisant une bonne adhésion des établissements et des élèves à la réforme.

\subsection{Évolution du taux de réussite dans les séries G et STEG}

\begin{figure}[ht]
\centering
\caption{Évolution du taux de réussite dans les séries G et STEG (2006-2024)}
\includegraphics[width=1\textwidth]{figure/taux_bac_STEG.png}
\label{fig:taux_reussite_STEG}
\end{figure}

La figure (Figure~\ref{fig:taux_reussite_STEG}) met en perspective les performances des deux séries en termes de réussite. 
On observe une forte variabilité du taux de réussite en série G, fluctuant généralement entre \textbf{40\%} et \textbf{50\%}, avec une tendance légèrement décroissante.

En revanche, la série STEG, dès sa première session en 2019, affiche \textbf{des taux de réussite supérieurs}, allant de \textbf{60\%} à plus de \textbf{75\% en 2024}.
Ce résultat semble valider l’objectif de la réforme qui est de renforcer l'efficacité du système en recentrant les contenus pédagogiques autour de compétences concrètes, professionnelles et transversales.

\bigskip

En somme, \textbf{la série STEG se distingue par de meilleures performances en matière de réussite}, tout en parvenant à capter un volume d’élèves au moins équivalent, voire supérieur, à celui de la série G à son apogée. 
Cela confirme la pertinence de la réforme dans le contexte de modernisation du système éducatif sénégalais.

\newpage
\section{analyse des séries Arabes et Franco-Arabes}

Le système éducatif sénégalais a progressivement intégré l'enseignement de l'arabe dans le secondaire, avec la création de séries Franco-arabes en 2000.
Cependant, face à la prolifération des établissements d'enseignement exclusivement arabe et des instituts islamiques, le décret n°2013-057 a modifié et complété ces dispositions a fin de mieux répondre à la demande sociale suivantes: 
\begin{itemize}
    \item Littératures et Civilisations arabes (L-AR)
    \item Mathématiques et Sciences physiques (S1-AR)
    \item Sciences expérimentales (S2-AR)
\end{itemize}

Les études secondaires arabes sont désormais sanctionnées par des diplômes appelés \textbf{baccalauréats arabes}, tandis que les enseignements bilingues franco-arabe donnent lieu à des \textbf{baccalauréats franco-arabes}.
\subsection{Série LA et LAR}

\begin{figure}[ht]
\centering
\caption{Évolution du nombre d'inscrits et du taux de réussite dans les séries LA et LAR (2006-2024)}
\includegraphics[width=1\textwidth]{figure/bac_LA_LAR.png}
\label{fig:LA_LAR}
\end{figure}

La Figure \ref{fig:LA_LAR} illustre l'évolution du nombre d'inscrits et des taux de réussite aux baccalauréats des séries Littératures et Sciences sociale (franco-arabe) LA et Littératures et Civilisations Arabes (L-AR) entre 2006 et 2024.

\subsubsection{Évolution du nombre d'inscrits}

La série LA, présente depuis 2000, a toujours eu un nombre d'inscrits relativement faible avec une légère croissance, passant de 10 en 2006 à 1 064 en 2024.  
Cette augmentation, bien que modérée, témoigne d'un intérêt croissant et soutenu pour cette filière au fil des ans.

En revanche, la série L-AR, introduite en 2013, a connu une croissance fulgurante dès son apparition. 
Elle a attiré 2 849 inscrits en 2013, atteignant un pic de 5 949 en 2015. Bien qu'une légère diminution ait été observée par la suite, le nombre d'inscrits en L-AR s'est stabilisé autour de 5 000 à 5 300, avec 5 184 inscrits en 2024. 
La série L-AR a rapidement dominé en termes d'effectifs, surpassant de loin la série LA, ce qui suggère une forte adhésion à cette nouvelle filière et un transfert significatif des étudiants vers cette option réformée.

Cette dynamique met en évidence la réussite de la mise en place du baccalauréat arabe, notamment la série L-AR, qui a su capter un volume important d'étudiants, répondant ainsi à la demande sociale et aux objectifs de structuration des études arabes au Sénégal.

\subsubsection{Évolution du taux de réussite}

La série LA a montré une grande variabilité de ses taux de réussite. Après un début à 55,6\% en 2006, elle a connu des pics remarquables à 90\% en 2008 et 91,7\% en 2010, des chiffres à juger avec prudence car l’effectif était considérablement faible, avec une moyenne d’environ 15 candidats entre 2006 et 2010.
Cependant, des baisses significatives ont suivi, avec des taux chutant à 46.7\% en 2012 et un point bas à 27.5\% en 2013. 
Par la suite, le taux de réussite en LA a montré une tendance à la reprise, atteignant 63.9\% en 2015 avant de se stabiliser autour de 50\% à 70\% dans les années suivantes, pour finir à 57.4\% en 2024. 

En comparaison, la série L-AR, bien que plus récente, a affiché des taux de réussite plus cohérents (explicables par un nombre de candidats assez élevé) avec une tendance croissante. 
À son introduction en 2013, le taux de réussite était d'environ 21\%, montant progressivement pour atteindre 55.5\% en 2024 qui se rapproche de celui de la série LA.
Ce résultat est d'autant plus notable que la série L-AR a géré un volume d'inscrits considérablement plus important.

En conclusion, la série L-AR, malgré un volume d'effectifs beaucoup plus important, a globalement réussi à maintenir des taux de réussite compétitifs et plus stables que ceux de la série LA.
Cette performance suggère que la réforme ayant introduit la série L-AR, avec son nouveau référentiel et ses objectifs d'harmonisation, a contribué à une meilleure efficacité du système pour les filières arabes au Sénégal.

\subsection{Série S2A}

\begin{figure}[ht]
\centering
\caption{Évolution du nombre d'inscrits et du taux de réussite dans la série S2A (2006-2024)}
\includegraphics[width=1\textwidth]{figure/bac_S2A.png}
\label{fig:S2A}
\end{figure}

La figure (Figure~\ref{fig:S2A}) montre l'évolution du nombre d'inscrits et du taux de réussite dans la série Sciences appliquées (S2A).

\subsubsection{Évolution du nombre d'inscrits}
De 2006 à 2016, la série S2A a connu une première phase de faible adhésion, avec une légère tendance à la hausse, passant de 4 inscrits en 2006 à 34 en 2016.
À partir de 2017, la série S2A enregistre une croissance exponentielle de ses effectifs, atteignant 262 inscrits en 2024.
La série parvient ainsi à capter un volume d’élèves en nette progression, traduisant une reconnaissance croissante de son importance.

\subsubsection{Évolution du taux de réussite :}
Concernant le taux de réussite, la série S2A présente une forte variabilité au début de la période, en raison du faible nombre de candidats. Après un taux de 50\% en 2006, il chute à 25\% en 2007, puis atteint un pic exceptionnel de 80\% en 2011.
À partir de 2016, on observe une nette amélioration, avec des taux de réussite généralement supérieurs à 60\%. La série atteint même 83,3\% en 2023, avant de légèrement reculer à 71,9\% en 2024.

\bigskip

Malgré des débuts irréguliers, la série S2A affiche de bonnes performances sur la dernière décennie, confirmant son potentiel de réussite croissant.

\subsection{Série S1A}

\begin{figure}[ht]
\centering
\caption{Évolution du nombre d'inscrits et du taux de réussite dans la série S2A (2006-2024)}
\includegraphics[width=1\textwidth]{figure/bac_S1A.png}
\label{fig:S1A}
\end{figure}

La figure (Figure~\ref{fig:S1A}) met en perspective l'évolution du nombre d'inscrits et du taux de réussite dans la série Sciences fondamentales (S1A).

\subsubsection{Évolution du nombre d'inscrits}

La série S1A se caractérise par un nombre d'inscrits extrêmement faible tout au long de la période étudiée. Introduite comme option du baccalauréat secondaire arabe, elle débute avec 2 inscrits et ne dépasse jamais 7 candidats. 
Cette faible adhésion, avec des années où les effectifs descendent à 1 ou 2, suggère que la série S1A est une filière attirant un public très très limité.

\subsubsection{Évolution du taux de réussite :}

En revanche, les performances de la série S1A en termes de taux de réussite sont remarquables. 
Malgré un nombre très restreint de candidats, le taux de réussite est de 100\% pour la grande majorité des années entre 2013 et 2024. 
La seule exception notable est l'année 2019, où le taux chute à 66.7\%. 
Cette régularité peut s’expliquer par une sélection rigoureuse des candidats.

\bigskip

En somme, la série S1A se distingue par un effectif très réduit, mais affiche une excellence constante en matière de réussite. 

\textbf{Remarque importante} : Les séries S1-AR (Mathématiques et Sciences physiques) et S2-AR (Sciences expérimentales), introduites par le décret de 2013 pour le baccalauréat arabe, n'ont jamais été concrètement mises en place ni organisées, comme mentionné dans le Chapitre 1.

\section{prédiction du taux de réussite}

\section{Conclusion} % Importation du chapitre 4

    \chapter{Analyse du parcours universitaire des bacheliers (UCAD)}

\section{Introduction}

Ce chapitre s'intéresse à la trajectoire des bacheliers, une fois admis à l'université, en particulier à l'UCAD. 
L’objectif est d’évaluer l’impact des différentes réformes du baccalauréat sur l’orientation et la réussite universitaire des étudiants issus des séries concernées.

Dans un premier temps, nous analyserons l’évolution des effectifs inscrits à l’UCAD selon les séries de baccalauréat impactées par les réformes.
Nous observerons ensuite les établissements et départements universitaires où ces bacheliers sont le plus souvent orientés, afin d’identifier les filières de destination privilégiées selon la série d’origine.

Enfin, une analyse de type \textbf{suivi de cohorte} permettra d’évaluer la progression et les performances académiques de ces étudiants dans le temps (réussite, redoublement, abandon), afin de mieux comprendre les effets des réformes sur la réussite universitaire.

\newpage
\section{Évolution des inscriptions à l'UCAD}

\subsection{Les inscrits des série STEG et G}

\begin{figure}[ht]
\centering
\caption{Évolution des inscriptions à l'UCAD pour les séries STEG et G}
\includegraphics[width=1\textwidth]{figure/Inscrits_ucad_STEG.png}
\label{fig:inscrits_ucad_steg}
\end{figure}

La figure (Figure~\ref{fig:inscrits_ucad_steg}) illustre l'évolution des inscriptions à l'Université Cheikh Anta Diop (UCAD) pour les bacheliers issus des séries G et STEG, sur la période 2010-2011 à 2023-2024.

Jusqu'à l'année universitaire 2018-2019, seule la série G est représentée à l'UCAD, avec un nombre d'inscrits fluctuant autour de 2 000 à 2 500. Un pic est observé en 2014-2015 avec 2 526 inscrits. 
À partir de 2019-2020, la série STEG fait son apparition, avec 725 inscrits, marquant le début de la transition. Simultanément, les effectifs de la série G commencent à décliner fortement, passant de 2 054 en 2019-2020 à seulement 115 en 2023-2024. 
Ce déclin correspond à la suppression progressive de la série G au profit de la série STEG dans le système du baccalauréat.

Le croisement des courbes est particulièrement visible en 2020-2021, où le nombre d'inscrits en STEG (1 347) dépasse celui de la série G (1 316). 
Cette tendance se confirme les années suivantes, la série STEG affichant des effectifs croissants (2 021 en 2021-2022, 2 026 en 2022-2023, et 2 016 en 2023-2024), tandis que la série G continue sa chute. 
La série STEG maintient ainsi un volume d'inscriptions à l'UCAD comparable à celui que la série G connaissait avant sa suppression, démontrant une transition quantitativement réussie au niveau de l'entrée à l'université. 
Ce transfert des effectifs de la série G vers la série STEG à l'UCAD est un indicateur de l'efficacité de la réforme du baccalauréat dans l'orientation des étudiants vers la nouvelle filière.

\newpage
\subsection{Les inscrits des séries Arables et Franco-Arabes}

\subsubsection{Série LA et LAR}

\begin{figure}[ht]
\centering
\caption{Évolution des inscriptions à l'UCAD pour les séries LA et LAR}
\includegraphics[width=1\textwidth]{figure/Inscrits_ucad_LA_LAR.png}
\label{fig:inscrits_ucad_la_lar}
\end{figure}

La figure (Figure~\ref{fig:inscrits_ucad_la_lar}) présente l'évolution des inscriptions à l'UCAD pour les bacheliers issus des séries Littératures Arabes (LA) et Littératures et Civilisations Arabes (L-AR) de 2010-2011 à 2023-2024.

La série LA, bien que présente depuis 2010-2011, a enregistré un nombre très faible d'inscrits à l'UCAD jusqu'en 2013-2014, oscillant entre 0 et 44. 
À partir de 2014-2015, le nombre d'inscrits en LA connaît une croissance progressive, passant de 50 à 671 en 2022-2023, avant de redescendre légèrement à 588 en 2023-2024.

La série L-AR, introduite à l'UCAD à partir de 2013-2014, a connu une croissance beaucoup plus rapide et significative. Elle débute avec 169 inscrits en 2013-2014 et grimpe rapidement pour atteindre un pic de 2 739 inscrits en 2021-2022. 
Bien qu'une légère diminution soit observée les années suivantes, avec 2 223 inscrits en 2022-2023 et 1 706 en 2023-2024, la série L-AR maintient un volume d'inscriptions considérablement plus élevé que la série LA. 
Ce phénomène s'aligne avec l'observation d'un transfert progressif des effectifs vers la série L-AR au niveau du baccalauréat lui-même, la série L-AR ayant été mise en place pour mieux encadrer et répondre à la demande sociale des filières arabes et franco-arabes. 

% En conclusion, la série L-AR a réussi à s'imposer comme la filière majeure pour les étudiants en études arabes à l'UCAD, absorbant une grande partie des effectifs et démontrant l'impact des réformes du baccalauréat sur l'orientation universitaire.

\newpage
\subsubsection{Série S2A et S1A}

\begin{figure}[ht]
\centering
\caption{Évolution des inscriptions à l'UCAD pour les séries S2A et S1A}
\includegraphics[width=1\textwidth]{figure/Inscrits_ucad_SA.png}
\label{fig:inscrits_ucad_sa}
\end{figure}

La figure (Figure~\ref{fig:inscrits_ucad_sa}) retrace l'évolution des inscriptions à l'UCAD pour les bacheliers des séries Sciences fondamentales (S1A) et Sciences appliquées (S2A) entre 2011-2012 et 2023-2024.

La série S1A, bien que présente, affiche un nombre d'inscrits extrêmement faible à l'UCAD tout au long de la période. Les effectifs oscillent entre 0 et 8 (atteint en 2019-2020), avec seulement 4 inscrits en 2023-2024. 
Cette quasi-absence d'inscriptions à l'université confirme le caractère très marginal de cette filière, qui déjà au niveau du baccalauréat, n'attire qu'un nombre dérisoire de candidats. 
Cela suggère que la série S1A ne débouche que sur très peu d'orientations universitaires à l'UCAD.

En revanche, la série S2A présente une dynamique d'inscriptions beaucoup plus significative. Après des débuts modestes entre 6 et 17 inscrits de 2011-2012 à 2015-2016, le nombre d'étudiants en S2A à l'UCAD connaît une croissance exponentielle. 
On passe de 25 inscrits en 2016-2017 à 107 en 2020-2021, pour atteindre un pic de 173 inscrits en 2023-2024. Cette forte augmentation des inscriptions en S2A à l'UCAD reflète une reconnaissance croissante de cette filière scientifique parmi les bacheliers arabes et franco-arabes, leur offrant des perspectives universitaires concrètes.

En somme, l'analyse des inscriptions à l'UCAD confirme les dynamiques observées au niveau du baccalauréat : la série S1A reste une filière confidentielle, tandis que la S2A gagne en importance et en attractivité pour les études supérieures, offrant ainsi une voie viable aux bacheliers issus de l'enseignement scientifique arabe et franco-arabe.

\newpage
\section{Répartition des Inscrits par Établissement et Département}

\subsection{Série STEG et G}

% \subsubsection{Série G}

\textbf{Répartition par Établissements}

\begin{figure}[ht]
\centering
\caption{Top 5 des établissements avec le plus d'inscrits (G, 2018-2019)}
\includegraphics[width=0.8\textwidth]{figure/etab_G_2019.png}
\label{fig:etab_g_2019}
\end{figure}

La Figure \ref{fig:etab_g_2019} présente la répartition des inscriptions des bacheliers G au sein des établissements de l'UCAD pour l'année universitaire 2018-2019.

Avant la réforme complète, en 2019, la Faculté des Sciences Économiques et de Gestion (FASEG) était déjà l'établissement accueillant la majorité des bacheliers de la série G, avec 1783 inscrits, suivie par l'École Supérieure Polytechnique (ESP) avec 524 inscrits. 

% \newpage
\begin{figure}[ht]
\centering
\caption{Top 5 des établissements avec le plus d'inscrits (STEG, 2023-2024)}
\includegraphics[width=0.8\textwidth]{figure/etab_STEG_2024.png}
\label{fig:etab_steg_2024}
\end{figure}

La Figure \ref{fig:etab_steg_2024} présente la répartition des inscriptions des bacheliers STEG au sein des établissements de l'UCAD pour l'année universitaire 2023-2024.

Pour l'année universitaire 2023-2024, après la transformation de la série G en STEG, la FASEG continue de dominer très largement les inscriptions des bacheliers STEG à l'UCAD, accueillant 1508 étudiants. 
L'École Supérieure Polytechnique (ESP) maintient sa deuxième position avec 493 inscrits. Ces chiffres confirment que, malgré le changement de dénomination et d'orientation pédagogique, la FASEG demeure la principale destination pour les bacheliers de cette filière, en raison de la nature économique et de gestion de la série STEG. 

% \newpage
\textbf{Répartition par Départements}

\begin{figure}[ht]
\centering
\caption{Top 5 des départements avec le plus d'inscrits (G, 2018-2019)}
\includegraphics[width=0.8\textwidth]{figure/dep_G_2019.png}
\label{fig:dep_g_2019}
\end{figure}

La Figure \ref{fig:dep_g_2019} illustre la répartition des inscrits G par département de formation à l'UCAD pour l'année universitaire 2018-2019.

En 2019, le département de Gestion concentrait la grande majorité des inscrits de la série G, avec 1835 étudiants, suivi par l'Institut de Formation en Administration et Création d'Entreprise avec 433 inscrits. Le département d'Économie comptait 35 inscrits.

% Le département de Gestion se positionne très largement en tête, avec 1 623 inscrits. Cette dominance est logique et attendue, étant donné l'orientation principale de la série G vers les sciences et techniques de gestion. 

\begin{figure}[ht]
\centering
\caption{Top 5 des départements avec le plus d'inscrits (STEG, 2023-2024)}
\includegraphics[width=0.8\textwidth]{figure/dep_STEG_2024.png}
\label{fig:dep_steg_2024}
\end{figure}

La Figure \ref{fig:dep_steg_2024} illustre la répartition des inscrits STEG par département de formation à l'UCAD pour l'année universitaire 2023-2024.

Pour l'année universitaire 2023-2024, le département de Gestion conserve sa position dominante pour les inscrits de la série STEG, avec 1623 inscrits. L'Institut de Formation en Administration et Création d'Entreprise suit avec 361 inscrits. 
Le département d'Économie compte 16 inscrits. Cette répartition des inscrits par département, avant et après la réforme, confirme la forte vocation de cette série pour les études en gestion, avec un intérêt secondaire pour l'entrepreneuriat, et une présence marginale dans les autres branches de l'économie. 

% \newpage
\subsection{Série Arabes et Franco-Arabes}

\subsubsection{Série LA}

\textbf{Établissements}

\begin{figure}[ht]
\centering
\caption{Top 5 des établissements avec le plus d'inscrits (LA, 2023-2024)}
\includegraphics[width=0.8\textwidth]{figure/etab_LA_2024.png}
\label{fig:etab_la_2024}
\end{figure}

La Figure \ref{fig:etab_la_2024} montre la répartition des inscrits de la série LA par établissement à l'UCAD pour l'année universitaire 2023-2024.

La Faculté des Lettres et Sciences Humaines (FLSH) est l'établissement qui accueille de loin le plus grand nombre de bacheliers LA, avec 508 inscrits. 
Cette prédominance est entièrement cohérente avec la nature littéraire de la série LA, la FLSH étant traditionnellement le pôle d'excellence pour les études de lettres, de langues et de civilisations.

La Faculté des Sciences Juridiques et Politiques (FSJP) arrive en deuxième position avec 65 inscrits. Bien que significativement moins importante que la FLSH, la présence de bacheliers LA dans cette faculté peut s'expliquer par un intérêt pour le droit islamique.

\newpage
\textbf{Départements}

\begin{figure}[ht]
\centering
\caption{Top 5 des départements avec le plus d'inscrits (LA, 2023-2024)}
\includegraphics[width=0.8\textwidth]{figure/dep_LA_2024.png}
\label{fig:dep_la_2024}
\end{figure}

La Figure \ref{fig:dep_la_2024} présente la répartition des inscrits de la série LA par département de formation à l'UCAD pour l'année universitaire 2023-2024.

Le département d'Arabe domine avec 207 bacheliers LA, ce qui est attendu pour cette série littéraire axée sur la langue arabe
Cependant, d’autres départements comme la Géographie (88), les Lettres Modernes (77), le Droit Privé (44) et l’Anglais (39) attirent aussi des bacheliers LA

Cette dispersion s’explique par le caractère franco-arabe de la série LA, qui offre plus de flexibilité que la série LAR, permettant aux étudiants de s’orienter vers un éventail plus large de filières en sciences humaines.
% \newpage
\subsubsection{Série LAR}

\textbf{Établissements}

\begin{figure}[ht]
\centering
\caption{Top 5 des établissements avec le plus d'inscrits (LAR, 2023-2024)}
\includegraphics[width=0.8\textwidth]{figure/etab_LAR_2024.png}
\label{fig:etab_lar_2024}
\end{figure}

La Figure \ref{fig:etab_lar_2024} illustre la répartition des inscrits de la série LAR par établissement à l'UCAD pour l'année universitaire 2023-2024.

Comme pour la série LA, la Faculté des Lettres et Sciences Humaines (FLSH) est l’établissement le plus plébiscité par les bacheliers LAR, avec un total massif de 1 616 inscrits. 
Cette écrasante majorité confirme le rôle central de la FLSH dans la formation en langue, littérature et civilisation arabes, faisant de cette faculté la destination naturelle et presque exclusive des bacheliers issus de cette série.

En deuxième position, la FASTEF accueille 78 inscrits. Cette orientation s'explique par les débouchés dans l’enseignement, en particulier pour les matières liées à la langue arabe ou à l’éducation islamique, domaines en cohérence avec la formation reçue dans la série LAR.

À l’inverse, la Faculté des Sciences Juridiques et Politiques (FSJP) ne compte que 2 inscrits LAR, contre 65 pour la série LA. 
Cette faible représentation s’explique très probablement par une contrainte linguistique importante : 
les enseignements à la FSJP se déroulant majoritairement en français, les bacheliers LAR, dont la formation est exclusivement en arabe, peuvent être freinés dans leur orientation vers des filières francophones, contrairement aux bacheliers LA disposant d’une double compétence linguistique.

\textbf{Départements}

\begin{figure}[ht]
\centering
\caption{Top 5 des départements avec le plus d'inscrits (LAR, 2023-2024)}
\includegraphics[width=0.8\textwidth]{figure/dep_LAR_2024.png}
\label{fig:dep_lar_2024}
\end{figure}

% La Figure \ref{fig:dep_lar_2024} détaille la répartition des inscrits de la série LAR par département de formation à l'UCAD pour l'année universitaire 2023-2024.

Le département d’Arabe domine de manière écrasante avec 1 680 inscrits, concentrant ainsi presque la totalité des bacheliers LAR. 
Cette prépondérance est tout à fait attendue, la série LAR (Littératures et Civilisations Arabes) étant spécifiquement conçue pour préparer les étudiants à des études approfondies en langue, littérature et culture arabes. 
Le département d’Arabe représente donc la destination la plus logique, naturelle et cohérente pour ces profils.

En conclusion, l’orientation des bacheliers LAR est quasiment exclusive vers le département d’Arabe, traduisant une spécialisation universitaire marquée, en parfaite continuité avec leur formation secondaire.

\subsubsection{Série S2A}

\textbf{Établissements}

\begin{figure}[ht]
\centering
\caption{Top 5 des établissements avec le plus d'inscrits (S2A, 2023-2024)}
\includegraphics[width=0.8\textwidth]{figure/etab_S2A_2024.png}
\label{fig:etab_s2a_2024}
\end{figure}

La Figure \ref{fig:etab_s2a_2024} présente la répartition des inscrits de la série S2A par établissement à l'UCAD pour l'année universitaire 2023-2024.

La FST domine avec 95 inscrits, reflet de l’orientation scientifique de la série S2A. 
D’autres établissements accueillent aussi ces étudiants, notamment la FASEG (38), la FMPO (19), ainsi que plus modestement l’ESP et l’IPP. Cette répartition illustre la polyvalence et l’adaptabilité des diplômés S2A, qui s’intègrent dans des filières variées allant des sciences à l’économie, la santé ou l’ingénierie.

\textbf{Départements}

\begin{figure}[ht]
\centering
\caption{Top 5 des départements avec le plus d'inscrits (S2A, 2023-2024)}
\includegraphics[width=0.8\textwidth]{figure/dep_S2A_2024.png}
\label{fig:dep_s2a_2024}
\end{figure}

La Figure \ref{fig:dep_s2a_2024} détaille la répartition des inscrits de la série S2A par département de formation à l'UCAD pour l'année universitaire 2023-2024.

Le département de Biologie Animale accueille le plus grand nombre de bacheliers S2A (45 inscrits), reflétant leur intérêt pour les sciences du vivant. 
Les départements de Mathématiques-Informatique (13) et de Médecine (11) enregistrent des effectifs plus modestes, montrant une ouverture vers des filières techniques et médicales. 
Cette diversité souligne la capacité des bacheliers S2A à s’intégrer dans des parcours variés grâce à leur formation polyvalente.

\newpage
\subsubsection{Série S1A}

\textbf{Établissements}

\begin{figure}[ht]
\centering
\caption{Top 5 des établissements avec le plus d'inscrits (S1A, 2023-2024)}
\includegraphics[width=0.8\textwidth]{figure/etab_S1A_2024.png}
\label{fig:etab_s1a_2024}
\end{figure}

La Figure \ref{fig:etab_s1a_2024} présente la répartition des inscrits de la série S1A par établissement à l'UCAD pour l'année universitaire 2023-2024.

La Faculté des Sciences et Technologies (FST) est la principale destination avec 3 inscrits, ce qui correspond à la nature fondamentale de la série S1A. 
L’École Supérieure Polytechnique (ESP) accueille un seul inscrit, reflétant un intérêt marginal pour les formations techniques.

Cependant, le très faible nombre d’inscrits dans ces établissements souligne la rareté de cette série à l’UCAD. 
Ces données limitées ne permettent pas de tirer de conclusions robustes sur les choix d’orientation des bacheliers S1A.

\textbf{Départements}

\begin{figure}[ht]
\centering
\caption{Top 5 des départements avec le plus d'inscrits (S1A, 2023-2024)}
\includegraphics[width=0.8\textwidth]{figure/dep_S1A_2024.png}
\label{fig:dep_s1a_2024}
\end{figure}

\newpage
La Figure \ref{fig:dep_s1a_2024} illustre la répartition des inscrits de la série S1A par département de formation à l'UCAD pour l'année universitaire 2023-2024.

Le département de Mathématiques-Informatique compte 3 inscrits, ce qui correspond bien au profil scientifique fondamental de la série S1A. Le Génie Électrique accueille 1 étudiant, reflétant un intérêt pour des applications techniques. 
Ces effectifs limités confirment la faible présence des bacheliers S1A à l’UCAD, qui restent néanmoins orientés vers des filières scientifiques et d’ingénierie spécialisées.

\newpage
\section{Analyse du parcours universitaire des bacheliers (suivi des cohortes)}

\subsection{Série Arabes et Franco-Arabes}
\subsubsection{cohorte 2018(LA)}

\begin{figure}[ht]
    \centering
    \caption{Graphe suivi de la cohorte 2018 (LA)}
    \includegraphics[width=1\textwidth]{figure/LA_2018.png}
\end{figure}

\newpage
\subsubsection{cohorte 2018(LAR)}

\begin{figure}[ht]
    \centering
    \caption{Graphe suivi de la cohorte 2018 (LAR)}
    \includegraphics[width=1\textwidth]{figure/LAR_2018.png}
\end{figure}

\newpage
\subsubsection{cohorte 2018(S2A)}

\begin{figure}[ht]
    \centering
    \caption{Graphe suivi de la cohorte 2018 (S2A)}
    \includegraphics[width=1\textwidth]{figure/S2A_2018.png}
\end{figure}

\newpage
\subsubsection{cohorte 2018(S1A)}

\begin{figure}[ht]
    \centering
    \caption{Graphe suivi de la cohorte 2018 (S1A)}
    \includegraphics[width=1\textwidth]{figure/S1A_2018.png}
\end{figure}

\newpage
\subsection{Série de reference pour les séries Arabes et Franco-Arabes}

\subsubsection{cohorte 2018(L'1)}

\begin{figure}[ht]
    \centering
    \caption{Graphe suivi de la cohorte 2018 (L'1)}
    \includegraphics[width=1\textwidth]{figure/L1_2018.png}
\end{figure}

\newpage
\subsubsection{cohorte 2018(S2)}

\begin{figure}[ht]
    \centering
    \caption{Graphe suivi de la cohorte 2018 (S2)}
    \includegraphics[width=1\textwidth]{figure/S2_2018.png}
\end{figure}

\newpage
\subsubsection{cohorte 2018(S1)}

\begin{figure}[ht]
    \centering
    \caption{Graphe suivi de la cohorte 2018 (S1)}
    \includegraphics[width=1\textwidth]{figure/S1_2018.png}
\end{figure}

\newpage
\subsection{Série STEG et G}

\subsubsection{cohorte 2019(STEG)}

\begin{figure}[ht]
    \centering
    \caption{Graphe suivi de la cohorte 2019 (STEG)}
    \includegraphics[width=1\textwidth]{figure/STEG_2019.png}
\end{figure}

\newpage
\subsubsection{cohorte 2018(G)}

\begin{figure}[ht]
    \centering
    \caption{Graphe suivi de la cohorte 2018 (G)}
    \includegraphics[width=1\textwidth]{figure/G_2018.png}
\end{figure}

\newpage

\section{Analyse des performances académiques}
\section{Conclusion} % Importation du chapitre 5

    \chapter{Restitution interactive et visualisation}

\section{Introduction}

Ce chapitre présente la phase finale de l’analyse : la restitution interactive des résultats à travers un tableau de bord conçu avec Power BI. 
L’objectif est de permettre une exploration dynamique et intuitive des données du baccalauréat sénégalais de 2006 à 2024, 
à destination notamment de l’Office du Baccalauréat et de toute personne souhaitant approfondir une étude sur le sujet.

\section{objectifs du dashboard}

Le tableau de bord a pour but :

\begin{itemize}
    \item de synthétiser les indicateurs clés liés au taux de réussite au baccalauréat par année, série et session ;
    \item d’observer l’évolution de ces indicateurs dans le temps ;
    \item de permettre un filtrage dynamique selon les besoins d’analyse ;
    \item et de soutenir la prise de décision à travers des visualisations claires et interactives.
\end{itemize}

\section{Présentation des indicateurs suivis}

Les indicateurs principaux intégrés dans le dashboard sont :
\begin{itemize}
    \item Le taux de réussite global par année ;
    \item Le taux de réussite par série ;
    \item Le nombre total d’admis et d’inscrits ;
    \item La répartition par session (1er ou 2e tour) ;
    \item La répartition par mention obtenue.
\end{itemize}

\section{Construction du dashboard Power BI}

\subsection{Première version : Résultats de 2024 uniquement}

Dans un premier temps, n’ayant pas encore accès à l’ensemble des données historiques, 
j’ai construit une première version du tableau de bord en me basant uniquement sur les résultats de l’année 2024. 
Cette version m’a permis de tester la structure du dashboard et de définir les visualisations pertinentes à suivre. 
Elle comportait déjà tous les indicateurs énumérés précédemment.
\begin{figure}[htbp]
    \centering
    \caption{Tableau de bord Power BI - Résultats du baccalauréat 2024}
    \includegraphics[width=15cm]{figure/bac_2024.png}
\end{figure}

\newpage
\subsection{ Deuxième version : Résultats de 2006 à 2024}

Une fois l’accès aux données complètes obtenu, j’ai pu enrichir le dashboard avec les résultats du bac de 2006 à 2024. 
Cette version permet d’analyser l’évolution temporelle des indicateurs clés et de mieux comprendre l’impact des différentes réformes au fil des années.
\begin{figure}[htbp]
    \centering
    \caption{Tableau de bord Power BI - Résultats du baccalauréat 2006 à 2024}
    \includegraphics[width=15cm]{figure/bac_2006_2024.png}
\end{figure}

\subsection{Utilisation de Power Query et création de mesures}

Power Query a été utilisé pour nettoyer et transformer les données importées dans Power BI. 
J’y ai notamment créé des colonnes conditionnelles pour catégoriser les mentions ou les sessions, 
et j’ai défini plusieurs mesures DAX (Data Analysis Expressions) telles que le taux de réussite, 
le nombre total d’inscrits ou encore les ratios d’admis par série.

\section{Conclusion}

La restitution interactive via Power BI offre une approche moderne et intuitive de l’analyse des données du baccalauréat. 
Elle facilite l’interprétation des tendances et appuie la prise de décision à partir d’une lecture visuelle et dynamique des résultats. 
Ce tableau de bord constitue un apport concret à l’Office du Baccalauréat, pouvant soutenir les futures études sur l’évolution du système éducatif au Sénégal. % Importation du chapitre 6

    \chapter{Discussion et Recomandation}

\section{Introduction}

\section{Conclusion} % Importation du chapitre 7
     
    \chapter*{Conclusion Générale}
\addcontentsline{toc}{chapter}{Conclusion Générale}

Ce mémoire a exploré l'impact multidimensionnel des récentes réformes du baccalauréat au Sénégal sur le taux de réussite à cet examen et sur le parcours universitaire des bacheliers à l'UCAD.

\section*{Synthèse des Contributions :}
\addcontentsline{toc}{section}{Synthèse des Contributions}

Nos principales contributions se situent à plusieurs niveaux :
\begin{enumerate}
    \item \textbf{Analyse Quantitative Approfondie :} Nous avons fourni une analyse détaillée de l'évolution des effectifs et des taux de réussite au baccalauréat sur une longue période (2006-2024), en identifiant les tendances générales et les points d'inflexion majeurs.
    \item \textbf{Impact des Réformes sur les Séries Spécifiques :} Nous avons mis en lumière le succès de la transition de la série G vers la STEG, caractérisée par une meilleure performance au bac. De même, nous avons démontré comment la réforme des filières arabes, notamment l'introduction de la série L-AR, a permis une croissance significative des effectifs et une meilleure structuration. La série S2A a également montré une dynamique positive.
    \item \textbf{Cartographie des Orientations Universitaires :} En analysant la répartition des bacheliers par établissement et département à l'UCAD, nous avons confirmé l'adéquation générale entre les séries du baccalauréat et les filières universitaires choisies, soulignant l'efficacité des réformes dans l'orientation des étudiants.
\end{enumerate}

\section*{Impact et Valeur Ajoutée :}
\addcontentsline{toc}{section}{Impact et Valeur Ajoutée}

Ce travail contribue au domaine de la Data Science appliquée à l'éducation en fournissant une analyse empirique basée sur des données réelles, ce qui est crucial pour la prise de décision politique. 
Il offre une évaluation concrète de l'efficacité des réformes éducatives, permettant aux décideurs de mieux comprendre les forces et les faiblesses du système. 
La modélisation de ces dynamiques offre des outils pour anticiper les flux d'étudiants et adapter l'offre universitaire. 
Les informations granulaires sur l'orientation peuvent aider à optimiser l'allocation des ressources dans les différents établissements et départements.

\section*{Contraintes et Limites :}
\addcontentsline{toc}{section}{Contraintes et Limites}

Il est crucial de reconnaître les contraintes rencontrées. 
L'un des défis majeurs dans l'analyse de l'impact des réformes est le facteur temps. Pour avoir une conclusion véritablement pertinente et robuste sur l'impact à long terme des réformes, notamment sur la réussite universitaire et l'insertion professionnelle, il serait nécessaire de disposer d'un recul temporel plus important. 
Les réformes les plus récentes n'ont eu que quelques années pour produire leurs effets, et l'impact complet sur le parcours universitaire (réussite en licence, master, insertion professionnelle) ne peut être pleinement mesuré qu'après plusieurs promotions complètes. 
Les données concernant les abandons ou les redoublements n'ont pas pu être intégrées en profondeur dans cette étude en raison de leur complexité et de la nécessité d'un suivi de cohorte détaillé sur plusieurs années.

\section*{Ouverture et Perspectives :}
\addcontentsline{toc}{section}{Ouverture et Perspectives}

Ce mémoire ouvre plusieurs pistes de recherche futures :
\begin{itemize}
    \item \textbf{Suivi de Cohorte Longitudinale :} Approfondir l'analyse du parcours universitaire par un suivi longitudinal des cohortes de bacheliers, en étudiant leur progression (réussite, redoublement, abandon) sur les cycles de licence et master.
    \item \textbf{Corrélation avec l'Insertion Professionnelle :} Évaluer l'adéquation entre les formations universitaires issues des nouvelles séries et les besoins du marché de l'emploi au Sénégal.
    \item \textbf{Analyse Qualitative Complémentaire :} Mener des entretiens avec les acteurs du système éducatif (enseignants, administrateurs), les bacheliers et les professionnels pour recueillir leurs perceptions des réformes.
    \item \textbf{Impact Socio-économique :} Analyser l'impact des réformes sur l'équité et l'accès à l'enseignement supérieur pour différentes catégories socio-économiques d'étudiants.
\end{itemize}


\section*{Recommandations pour les Réformes Futures :}
\addcontentsline{toc}{section}{Recommandations pour les Réformes Futures}

Au vu de cette analyse, plusieurs recommandations peuvent être formulées pour les réformes futures :
\begin{enumerate}
    \item \textbf{Consolider les Acquis des Réformes :} Poursuivre le renforcement des filières qui ont montré leur efficacité (STEG, L-AR, S2A) en assurant un alignement continu des programmes du secondaire avec les exigences de l'enseignement supérieur et du marché du travail.
    \item \textbf{Réévaluer les Filières Marginales :} Une réflexion approfondie est nécessaire concernant la série S1A et les filières S1-AR/S2-AR non implémentées. Il est essentiel de comprendre pourquoi ces filières n'attirent pas ou n'ont pas été développées, afin de les adapter, les fusionner ou les repenser si elles répondent à un besoin non satisfait.
    \item \textbf{Investir dans le Suivi des Étudiants : }Mettre en place des systèmes robustes de collecte de données longitudinales sur le parcours universitaire et l'insertion professionnelle des bacheliers. Ces données sont cruciales pour une évaluation continue et l'ajustement des politiques éducatives.
    \item \textbf{Préparer l'Université aux Nouveaux Flux :} Avec l'augmentation et la redistribution des effectifs du baccalauréat, il est impératif d'anticiper les besoins en infrastructures, en personnel enseignant et en ressources pédagogiques à l'UCAD et dans les autres établissements d'enseignement supérieur.
    \item \textbf{Communication et Orientation :} Renforcer les dispositifs d'information et d'orientation des élèves du secondaire sur les nouvelles séries du baccalauréat et les débouchés universitaires et professionnels associés, afin d'assurer des choix éclairés et une meilleure adéquation entre les profils et les filières.
\end{enumerate} % Importation du chapitre 3

    \printbibliography % Impression de la bibliographie
\end{document}