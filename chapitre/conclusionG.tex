\chapter*{Conclusion Générale}
\addcontentsline{toc}{chapter}{Conclusion Générale}

Ce mémoire a exploré l'impact multidimensionnel des récentes réformes du baccalauréat au Sénégal sur le taux de réussite à cet examen et sur le parcours universitaire des bacheliers à l'UCAD.

\section*{Synthèse des Contributions :}
\addcontentsline{toc}{section}{Synthèse des Contributions}

Nos principales contributions se situent à plusieurs niveaux :
\begin{enumerate}
    \item \textbf{Analyse Quantitative Approfondie :} Nous avons fourni une analyse détaillée de l'évolution des effectifs et des taux de réussite au baccalauréat sur une longue période (2006-2024), en identifiant les tendances générales et les points d'inflexion majeurs.
    \item \textbf{Impact des Réformes sur les Séries Spécifiques :} Nous avons mis en lumière le succès de la transition de la série G vers la STEG, caractérisée par une meilleure performance au bac. De même, nous avons démontré comment la réforme des filières arabes, notamment l'introduction de la série L-AR, a permis une croissance significative des effectifs et une meilleure structuration. La série S2A a également montré une dynamique positive.
    \item \textbf{Cartographie des Orientations Universitaires :} En analysant la répartition des bacheliers par établissement et département à l'UCAD, nous avons confirmé l'adéquation générale entre les séries du baccalauréat et les filières universitaires choisies, soulignant l'efficacité des réformes dans l'orientation des étudiants.
\end{enumerate}

\section*{Impact et Valeur Ajoutée :}
\addcontentsline{toc}{section}{Impact et Valeur Ajoutée}

Ce travail contribue au domaine de la Data Science appliquée à l'éducation en fournissant une analyse empirique basée sur des données réelles, ce qui est crucial pour la prise de décision politique. 
Il offre une évaluation concrète de l'efficacité des réformes éducatives, permettant aux décideurs de mieux comprendre les forces et les faiblesses du système. 
La modélisation de ces dynamiques offre des outils pour anticiper les flux d'étudiants et adapter l'offre universitaire. 
Les informations granulaires sur l'orientation peuvent aider à optimiser l'allocation des ressources dans les différents établissements et départements.

\section*{Contraintes et Limites :}
\addcontentsline{toc}{section}{Contraintes et Limites}

Il est crucial de reconnaître les contraintes rencontrées. 
L'un des défis majeurs dans l'analyse de l'impact des réformes est le facteur temps. Pour avoir une conclusion véritablement pertinente et robuste sur l'impact à long terme des réformes, notamment sur la réussite universitaire et l'insertion professionnelle, il serait nécessaire de disposer d'un recul temporel plus important. 
Les réformes les plus récentes n'ont eu que quelques années pour produire leurs effets, et l'impact complet sur le parcours universitaire (réussite en licence, master, insertion professionnelle) ne peut être pleinement mesuré qu'après plusieurs promotions complètes. 
Les données concernant les abandons ou les redoublements n'ont pas pu être intégrées en profondeur dans cette étude en raison de leur complexité et de la nécessité d'un suivi de cohorte détaillé sur plusieurs années.

\section*{Ouverture et Perspectives :}
\addcontentsline{toc}{section}{Ouverture et Perspectives}

Ce mémoire ouvre plusieurs pistes de recherche futures :
\begin{itemize}
    \item \textbf{Suivi de Cohorte Longitudinale :} Approfondir l'analyse du parcours universitaire par un suivi longitudinal des cohortes de bacheliers, en étudiant leur progression (réussite, redoublement, abandon) sur les cycles de licence et master.
    \item \textbf{Corrélation avec l'Insertion Professionnelle :} Évaluer l'adéquation entre les formations universitaires issues des nouvelles séries et les besoins du marché de l'emploi au Sénégal.
    \item \textbf{Analyse Qualitative Complémentaire :} Mener des entretiens avec les acteurs du système éducatif (enseignants, administrateurs), les bacheliers et les professionnels pour recueillir leurs perceptions des réformes.
    \item \textbf{Impact Socio-économique :} Analyser l'impact des réformes sur l'équité et l'accès à l'enseignement supérieur pour différentes catégories socio-économiques d'étudiants.
\end{itemize}


\section*{Recommandations pour les Réformes Futures :}
\addcontentsline{toc}{section}{Recommandations pour les Réformes Futures}

Au vu de cette analyse, plusieurs recommandations peuvent être formulées pour les réformes futures :
\begin{enumerate}
    \item \textbf{Consolider les Acquis des Réformes :} Poursuivre le renforcement des filières qui ont montré leur efficacité (STEG, L-AR, S2A) en assurant un alignement continu des programmes du secondaire avec les exigences de l'enseignement supérieur et du marché du travail.
    \item \textbf{Réévaluer les Filières Marginales :} Une réflexion approfondie est nécessaire concernant la série S1A et les filières S1-AR/S2-AR non implémentées. Il est essentiel de comprendre pourquoi ces filières n'attirent pas ou n'ont pas été développées, afin de les adapter, les fusionner ou les repenser si elles répondent à un besoin non satisfait.
    \item \textbf{Investir dans le Suivi des Étudiants : }Mettre en place des systèmes robustes de collecte de données longitudinales sur le parcours universitaire et l'insertion professionnelle des bacheliers. Ces données sont cruciales pour une évaluation continue et l'ajustement des politiques éducatives.
    \item \textbf{Préparer l'Université aux Nouveaux Flux :} Avec l'augmentation et la redistribution des effectifs du baccalauréat, il est impératif d'anticiper les besoins en infrastructures, en personnel enseignant et en ressources pédagogiques à l'UCAD et dans les autres établissements d'enseignement supérieur.
    \item \textbf{Communication et Orientation :} Renforcer les dispositifs d'information et d'orientation des élèves du secondaire sur les nouvelles séries du baccalauréat et les débouchés universitaires et professionnels associés, afin d'assurer des choix éclairés et une meilleure adéquation entre les profils et les filières.
\end{enumerate}