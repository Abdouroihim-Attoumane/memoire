\chapter{Nettoyage et préparation des données}

\section{Introduction}
Avant d’entamer toute analyse statistique, une étape cruciale consiste à examiner, nettoyer et structurer les données. 
Bien que les jeux de données utilisés dans ce projet proviennent de sources officielles telles que l’Office du Bac et la DISI/UCAD, et présentent globalement une structure cohérente, un travail de nettoyage s’est avéré nécessaire.

Contrairement à l’hypothèse initiale selon laquelle les données ne nécessiteraient qu’un traitement minimal, plusieurs opérations classiques de nettoyage ont finalement été réalisées. 
Cela inclut la détection et le traitement de valeurs aberrantes, la gestion des doublons, ainsi que le traitement ciblé des valeurs manquantes. 
  Par exemple, dans les résultats universitaires, certaines valeurs nulles traduisent l'absence à une évaluation, mais d'autres relevaient d’incohérences ou de saisies incomplètes qu’il a fallu corriger ou exclure selon le contexte.

En parallèle, des opérations de préparation ont été menées pour structurer les données en vue de l’analyse. 
Celles-ci comprennent la fusion de bases de données complémentaires (Les Inscrits et Résultats de UCAD), la standardisation des formats de certaines variables, ainsi que la création de nouvelles colonnes dérivées utiles à l’étude.

Les sous-sections suivantes détaillent les étapes spécifiques effectuées lors de ce processus de nettoyage et de préparation.


\newpage
\section{Préparation de données des résultats du baccalauréat}

Les données relatives aux résultats du baccalauréat ont été centralisées dans une table unique contenant plus de deux millions d’enregistrements, couvrant la période de 2006 à 2024. 
Le code ci-dessous (~\ref{lst:info_bac}) donne un aperçu général de la structure du DataFrame après concaténation des fichiers annuels.
\begin{lstlisting}[language=Python,
    caption=Informations général du DataFrame,
    label=lst:info_bac,
    basicstyle=\ttfamily\small,
    backgroundcolor=\color{gray!10}
]
<class 'pandas.core.frame.DataFrame'>
RangeIndex: 2236490 entries, 0 to 2236489
Data columns (total 15 columns):
 #   Column           Dtype 
---  ------           ----- 
 0   nom              object
 1   prenom           object
 2   numero_table     object
 3   serie            object
 4   sexe             object
 5   age              object
 6   etablissement    object
 7   type_candidat    object
 8   resultat         object
 9   acad_provenance  object
 10  moy_finale       object
 11  mention          object
 12  abs              object
 13  exclusion        object
 14  year             object
dtypes: object(15)
memory usage: 255.9+ MB
\end{lstlisting}

On y retrouve des informations telles que le prénom, le nom, le numéro de table, la série, le sexe, l’âge, l'établissement, les résultats, la moyenne finale, la mention, etc. 
On observe un total de 14 colonnes et 2236490 lignes.
\newpage
\subsection{Doublons dans la clé primaire \texttt{numero\_table}}

\begin{lstlisting}[language=Python,
    caption=Nombre de numero\_table en doublon par année,
    label=lst:doublons,
    basicstyle=\ttfamily\small,
    backgroundcolor=\color{gray!10}
]
Annee 2006 : 0 numeros apparaissent au moins deux fois
Annee 2007 : 0 numeros apparaissent au moins deux fois
Annee 2008 : 371 numeros apparaissent au moins deux fois
Annee 2009 : 0 numeros apparaissent au moins deux fois
Annee 2010 : 0 numeros apparaissent au moins deux fois
Annee 2011 : 1 numeros apparaissent au moins deux fois
Annee 2012 : 2 numeros apparaissent au moins deux fois
Annee 2013 : 0 numeros apparaissent au moins deux fois
Annee 2014 : 0 numeros apparaissent au moins deux fois
Annee 2015 : 0 numeros apparaissent au moins deux fois
Annee 2016 : 0 numeros apparaissent au moins deux fois
Annee 2017 : 0 numeros apparaissent au moins deux fois
Annee 2018 : 0 numeros apparaissent au moins deux fois
Annee 2019 : 0 numeros apparaissent au moins deux fois
Annee 2020 : 0 numeros apparaissent au moins deux fois
Annee 2021 : 0 numeros apparaissent au moins deux fois
Annee 2022 : 0 numeros apparaissent au moins deux fois
Annee 2023 : 0 numeros apparaissent au moins deux fois
Annee 2024 : 0 numeros apparaissent au moins deux fois 
\end{lstlisting}

Les résultats révèlent que la majorité des années ne présentent aucun doublon. Toutefois, quelques
années, comme 2008 (371 doublons), 2011 (1) et 2012 (2) comportent des cas où deux
candidats partagent le même numéro de table dans la même année. Cela peut s’expliquer par des
erreurs de saisie ou des anomalies administratives.

\begin{table}[h]
\scriptsize
\centering
\caption{lignes avec les numéros en doublon en 2012}
\label{tab:doublon}
\begin{tabular}{llllllr}
\toprule
nom & numero\_table & serie & sexe & age & etablissement & moy\_finale \\
\midrule
CISSE & 56234 & L'1 & M & 17 & LYCEE CHARLES DEGAULLE & 11,00 \\
KANE & 56234 & L2 & M & 21 & LYCEE DE ROSS - BETHIO & 07,07 \\
BOYE & 56513 & L'1 & F & 19 & LYCEE EL HADJ OMAR FOUTIYOU TALL & 05,58 \\
CAMARA & 56513 & L'1 & M & 23 & STRUCTURE D'ENTRE AIDE DU LYCEE EL HADJ OMAR TALL & 07,17 \\
\bottomrule
\end{tabular}
\end{table}

Le tableau (table~\ref{lst:doublon}) illustre un exemple concret pour l’année 2012, où deux élèves de centres différents ont le même numéro de table mais des informations distinctes. 
Ces cas ont été traités avec prudence dans la suite des analyses.

\textbf{Remarque :}

On remarque qu’il ne s’agit pas de doublons exacts, mais de personnes différentes à qui le même numéro de table a été attribué.

\newpage
\subsection{Les valeurs manquantes dans les colonnes}

\begin{table}[h]
\hspace{5cm}
\caption{Valeurs manquantes dans les colonnes du DataFrame des résultats du bac}
\begin{tabular}{lr}
\toprule
Colonnes & nb\_valeur\_null \\
\midrule
nom & 1 \\
prenom & 1 \\
numero\_table & 3 \\
serie & 5442 \\
sexe & 1 \\
age & 1 \\
etablissement & 1 \\
type\_candidat & 1 \\
resultat & 3977 \\
acad\_provenance & 1 \\
moy\_finale & 11 \\
mention & 945542 \\
abs & 1 \\
exclusion & 1 \\
year & 0 \\
\bottomrule
\end{tabular}
\end{table}

\subsubsection{Valeurs manquantes dans la colonne \texttt{moy\_finale}}

\begin{table}[h]
\hspace{5cm}
\caption{Valeurs manquantes dans la colonne moy\_finale}
\begin{tabular}{lrllr}
\toprule
nom & numero\_table & resultat & moy\_finale \\
\midrule
NaN & 42379 & NaN & NaN \\
XXXXXXXX & 41803 & NaN & NaN \\
XXXXXXXX & 42043 & NaN & NaN \\
XXXXXXXX & 27911 & NaN & NaN \\
XXXXXXXX & 24635 & NaN & NaN \\
XXXXXXXX & 17167 & NaN & NaN \\
XXXXXXXX & 17669 & NaN & NaN \\
XXXXXXXX & 1054 & NaN & NaN \\
NDIAYE & NaN & NaN & NaN \\
FAYE & NaN & NaN & NaN \\
KANE & NaN & NaN & NaN \\
\bottomrule
\end{tabular}
\end{table}

\subsubsection{Valeurs manquantes dans la colonne \texttt{resultat}}

\textbf{Les valeurs possibles dans la colonne \texttt{resultat}}

\newpage
\subsection{Correction du type de la colonne \texttt{moy\_finale}}

Comme on peut le constater dans le code présenté (listing~\ref{lst:info_bac}), le type initial de la colonne \texttt{moy\_finale} n'était pas exploitable tel quel. 
En effet, certaines valeurs contenaient des tirets, d'autres utilisaient des virgules comme séparateur décimal, et certaines étaient tout simplement vides ou non numériques.

\begin{lstlisting}[language=Python,
    caption=Correction du type de la colonne moy\_finale,
    label=lst:moy_finale_type,
    basicstyle=\ttfamily\small,
    backgroundcolor=\color{gray!10}
]
all_data_filtre['moy_finale'] = (
    all_data_filtre['moy_finale']
    .astype(str) # Convertir en chaine de caracteres
    .str.replace('-', '', regex=False) # Supprimer les tirets
    .str.replace(',', '.', regex=False) # les virgules en points
    .str.strip() # Supprimer les espaces au debut et fin de chaine
    .replace({'': None, 'nan': None}) # Supprimer les chaines vides
    .astype(float) # Convertir en float
)
\end{lstlisting}

Pour rendre cette colonne exploitable statistiquement, plusieurs opérations de nettoyage ont été effectuées :
\begin{itemize}
\item Suppression des tirets \texttt{'-'}.
\item Remplacement des virgules \texttt{','} par des points \texttt{'.'}.
\item Suppression des espaces superflus.
\item Conversion des chaînes vides ou non valides (\texttt{''}, \texttt{'nan'}) en valeurs manquantes.
\item Conversion finale de la colonne en type \texttt{float}.
\end{itemize}

Ces étapes garantissent la cohérence de la variable \texttt{moy\_finale} pour les analyses statistiques à venir.

\newpage
\subsection{Création des colonnes \texttt{admis} et \texttt{session}}

Pour faciliter les analyses, deux nouvelles colonnes ont été dérivées de la variable \texttt{resultat} :
\begin{itemize}
\item La colonne \texttt{admis}, indiquant si un candidat est admis ou non. Les codes \texttt{'111'} et \texttt{'101'} ont été interprétés comme signifiant « admis ».
\item La colonne \texttt{session}, précisant s’il s’agit du premier ou du second tour, en se basant sur les mêmes codes.
\end{itemize}

\begin{lstlisting}[language=Python,
    caption=Création de nouvelles colonnes,
    label=lst:creation_colonnes,
    basicstyle=\ttfamily\small,
    backgroundcolor=\color{gray!10}
]
# Creation de la colonne 'admis' de 'resultat'
all_data['admis'] = all_data['resultat'].apply(lambda x: 'admis' 
                            if pd.notna(x) and x in ['111', '101'] 
                            else 'non admis')

# Creation de la colonne 'session' de 'admis'
all_data['session'] = all_data['resultat'].apply(lambda x: '1er Tour' 
                            if pd.notna(x) and x == '111' 
                            else ('2e Tour' if pd.notna(x) and x == '101' 
                                    else ''))
\end{lstlisting}

Ce pré-traitement permet d’améliorer la lisibilité des résultats, notamment dans les agrégations et visualisations.

\newpage
\section{Fusion des données des Inscrits et des Résultats de l’UCAD}

\subsection{Données des inscriptions à l’UCAD}

Les données d’inscription utilisées dans cette étude couvrent la période de 2002 à 2024. 
Cependant, pour garantir la cohérence avec les données de résultats (disponibles uniquement de 2011 à 2024), 
nous avons retenu uniquement les inscriptions allant de 2011 à 2023.

La sortie de code \ref{lst:inscription_ucad} présente un aperçu global de la base de données des inscriptions à l’UCAD pour les années universitaires allant de 2011 à 2023. 
Elle contient 1 141 120 enregistrements répartis sur 20 variables. Les variables \textit{NUMERO} et \textit{ANNEE UNIVERSITAIRE}, 
constituent les clés essentielles pour la fusion avec la base des résultats académiques.


\begin{lstlisting}[language=Python,
    caption=Info global du data des inscriptions, 
    label=lst:inscription_ucad, 
    basicstyle=\ttfamily\footnotesize, 
    backgroundcolor=\color{gray!10}
]
<class 'pandas.core.frame.DataFrame'>
RangeIndex: 1141120 entries, 0 to 1141119
Data columns (total 14 columns):
 #   Column                 Non-Null Count    Dtype  
---  ------                 --------------    -----  
 0   NUMERO                 1141120 non-null  object 
 1   SEXE                   1141120 non-null  object 
 2   ANNEE_BACC             1136174 non-null  float64
 3   NATIONALITE            1141120 non-null  object 
 4   SERIE_BACC             1109114 non-null  object 
 5   ETABLISSMENT_CODE      1141120 non-null  object 
 6   NIVEAU_SECTION         1141120 non-null  object 
 7   ANNEE_INSCRIPTION      1141120 non-null  int64  
 8   ANNEE_UNIVERSITAIRE    1141120 non-null  object 
 9   TYPE_FORMATION         1141120 non-null  object 
 10  CODE_NIVEAU            1141120 non-null  int64  
 11  NIVEAU LMD ET NON LMD  1141120 non-null  object 
 12  SYSTEME                1141120 non-null  object 
 13  DEPARTEMENT FORMATION  1141120 non-null  object 
dtypes: float64(1), int64(2), object(11)
memory usage: 121.9+ MB
\end{lstlisting}

\newpage
\subsection{Données des résultats de l’UCAD}

La sortie de code \ref{lst:resultat} présente les informations générales de la base de données contenant les résultats universitaires des étudiants de l’UCAD. 
Cette base couvre la période allant de 2010 à 2024, et concerne uniquement les établissements ayant effectué leurs délibérations sur la plateforme institutionnelle de la DISI. 
On y retrouve un total de 753 828 enregistrements répartis sur 27 colonnes.

\begin{lstlisting}[language=Python,
    caption=Info global du data des résultats, 
    label=lst:resultat_ucad, 
    basicstyle=\ttfamily\footnotesize, 
    backgroundcolor=\color{gray!10}
]
<class 'pandas.core.frame.DataFrame'>
RangeIndex: 753693 entries, 0 to 753692
Data columns (total 3 columns):
 #   Column               Non-Null Count   Dtype 
---  ------               --------------   ----- 
 0   NUMERO               753693 non-null  object
 1   ANNEE UNIVERSITAIRE  753693 non-null  object
 2   RESULTAT             753562 non-null  object
dtypes: object(3)
memory usage: 17.3+ MB
\end{lstlisting}

\newpage
\subsection{Fusion des données d’inscription et de résultats}

Pour fusionner les bases de données d’inscription et de résultats, 
les colonnes \textit{NUMERO} et \textit{ANNEE UNIVERSITAIRE} ont été utilisées comme clés de jointure. 

\begin{lstlisting}[language=Python,
    caption=Jointure des données d’inscription et de résultats,
    label=lst:jointure_ucad,
    basicstyle=\ttfamily\small,
    backgroundcolor=\color{gray!10}
]
df_final = pd.merge(df_inscrit, # Premier DataFrame
                df_resultat, # Deuxieme DataFrame
                on=['NUMERO', 'ANNEE UNIVERSITAIRE'], # Cles de jointure
                how='left' # garde toutes les lignes de df_inscrit
                ) 
\end{lstlisting}

La fusion a été réalisée à l’aide de l’option \texttt{how='left'} pour conserver l’ensemble des inscrits, 
y compris ceux pour lesquels aucun résultat académique n’est disponible.

\begin{lstlisting}[language=Python,
    caption=Info global du data des inscriptions et résultats, 
    label=lst:inscription_resultat_ucad, 
    basicstyle=\ttfamily\footnotesize, 
    backgroundcolor=\color{gray!10}
]
<class 'pandas.core.frame.DataFrame'>
RangeIndex: 1215837 entries, 0 to 1215836
Data columns (total 15 columns):
 #   Column                 Non-Null Count    Dtype  
---  ------                 --------------    -----  
 0   NUMERO                 1215837 non-null  object 
 1   SEXE                   1215837 non-null  object 
 2   ANNEE_BACC             1210891 non-null  float64
 3   NATIONALITE            1215837 non-null  object 
 4   SERIE_BACC             1183525 non-null  object 
 5   ETABLISSMENT_CODE      1215837 non-null  object 
 6   NIVEAU_SECTION         1215837 non-null  object 
 7   ANNEE_INSCRIPTION      1215837 non-null  int64  
 8   ANNEE UNIVERSITAIRE    1215837 non-null  object 
 9   TYPE_FORMATION         1215837 non-null  object 
 10  CODE_NIVEAU            1215837 non-null  int64  
 11  NIVEAU LMD ET NON LMD  1215837 non-null  object 
 12  SYSTEME                1215837 non-null  object 
 13  DEPARTEMENT FORMATION  1215837 non-null  object 
 14  RESULTAT               750198 non-null   object 
dtypes: float64(1), int64(2), object(12)
memory usage: 139.1+ MB
\end{lstlisting}

Cela explique la présence d’un grand nombre de valeurs manquantes dans la colonne \textit{RESULTAT}, 
notamment pour les étudiants dont les résultats n'ont pas encore été délibérés ou publiés sur la plateforme.

\newpage

\section{Filtage des données}
\section{Conclusion}