\chapter{Nettoyage et préparation des données}

\section{Introduction}
Avant d’entamer toute analyse statistique, une étape cruciale consiste à examiner, nettoyer et structurer les données. 
Bien que les jeux de données utilisés dans ce projet proviennent de sources officielles telles que l’Office du Bac et la DISI/UCAD, et présentent globalement une structure cohérente, un travail de nettoyage s’est avéré nécessaire.

Contrairement à l’hypothèse initiale selon laquelle les données ne nécessiteraient qu’un traitement minimal, plusieurs opérations classiques de nettoyage ont finalement été réalisées. 
Cela inclut la détection et le traitement de valeurs aberrantes, la gestion des doublons, ainsi que le traitement ciblé des valeurs manquantes. 
  Par exemple, dans les résultats universitaires, certaines valeurs nulles traduisent l'absence à une évaluation, mais d'autres relevaient d’incohérences ou de saisies incomplètes qu’il a fallu corriger ou exclure selon le contexte.

En parallèle, des opérations de préparation ont été menées pour structurer les données en vue de l’analyse. 
Celles-ci comprennent la fusion de bases de données complémentaires (Les Inscrits et Résultats de UCAD), la standardisation des formats de certaines variables, ainsi que la création de nouvelles colonnes dérivées utiles à l’étude.

% Les sous-sections suivantes détaillent les étapes spécifiques effectuées lors de ce processus de nettoyage et de préparation.

\section{Préparation de données des résultats du baccalauréat}

Les données relatives aux résultats du baccalauréat ont été consolidées dans une table unique après concaténation de fichiers annuels couvrant la période de 2006 à 2024. 
Cette base contient plus de deux millions d’enregistrements, représentant une volumétrie significative pour l’analyse.

Le code ci-dessous (code~\ref{lst:info_bac}) donne un aperçu général de la structure du DataFrame après concaténation des fichiers annuels.
\begin{lstlisting}[language=Python,
    caption=Informations général du DataFrame,
    label=lst:info_bac,
    basicstyle=\ttfamily\small,
    backgroundcolor=\color{gray!10}
]
 <class 'pandas.core.frame.DataFrame'>
 RangeIndex: 2236490 entries, 0 to 2236489
 Data columns (total 15 columns):
  #   Column           Dtype 
 ---  ------           ----- 
  0   nom              object
  1   prenom           object
  2   numero_table     object
  3   serie            object
  4   sexe             object
  5   age              object
  6   etablissement    object
  7   type_candidat    object
  8   resultat         object
  9   acad_provenance  object
  10  moy_finale       object
  11  mention          object
  12  abs              object
  13  exclusion        object
  14  year             object
 dtypes: object(15)
 memory usage: 255.9+ MB
\end{lstlisting}

Cette base comprend des informations variées telles que le nom, le prénom, le numéro de table, la série, le sexe, l’âge, l’établissement d’origine, le type de candidat, les résultats obtenus, la mention, la moyenne finale, ainsi que l’année de passage. 
Elle constitue une source clé pour l’analyse du taux de réussite et l’évaluation de l’impact des réformes introduites dans certaines séries.

\subsection{Doublons dans la clé primaire \texttt{numero\_table}}

L’identifiant principal des candidats étant le \texttt{numero\_table}, une vérification a été effectuée afin d’identifier d’éventuels doublons par année.

\newpage
\begin{lstlisting}[language=Python,
    caption=Nombre de numero\_table en doublon par année,
    label=lst:doublons,
    basicstyle=\ttfamily\small,
    backgroundcolor=\color{gray!10}
]
 Annee 2006 : 0 numeros apparaissent au moins deux fois
 Annee 2007 : 0 numeros apparaissent au moins deux fois
 Annee 2008 : 371 numeros apparaissent au moins deux fois
 Annee 2009 : 0 numeros apparaissent au moins deux fois
 Annee 2010 : 0 numeros apparaissent au moins deux fois
 Annee 2011 : 1 numeros apparaissent au moins deux fois
 Annee 2012 : 2 numeros apparaissent au moins deux fois
 Annee 2013 : 0 numeros apparaissent au moins deux fois
 Annee 2014 : 0 numeros apparaissent au moins deux fois
 Annee 2015 : 0 numeros apparaissent au moins deux fois
 Annee 2016 : 0 numeros apparaissent au moins deux fois
 Annee 2017 : 0 numeros apparaissent au moins deux fois
 Annee 2018 : 0 numeros apparaissent au moins deux fois
 Annee 2019 : 0 numeros apparaissent au moins deux fois
 Annee 2020 : 0 numeros apparaissent au moins deux fois
 Annee 2021 : 0 numeros apparaissent au moins deux fois
 Annee 2022 : 0 numeros apparaissent au moins deux fois
 Annee 2023 : 0 numeros apparaissent au moins deux fois
 Annee 2024 : 0 numeros apparaissent au moins deux fois 
\end{lstlisting}

Les résultats révèlent que la majorité des années ne présentent aucun doublon. 
Toutefois, quelques années, comme 2008 (371 doublons), 2011 (1) et 2012 (2) comportent des cas où deux candidats partagent le même numéro de table dans la même année.

\begin{table}[ht]
\scriptsize
\centering
\caption{lignes avec les numéros en doublon en 2012}
\label{tab:doublon}
\begin{tabular}{llllllr}
\toprule
nom & numero\_table & serie & sexe & age & etablissement & moy\_finale \\
\midrule
CISSE & 56234 & L'1 & M & 17 & LYCEE CHARLES DEGAULLE & 11,00 \\
KANE & 56234 & L2 & M & 21 & LYCEE DE ROSS - BETHIO & 07,07 \\
BOYE & 56513 & L'1 & F & 19 & LYCEE EL HADJ OMAR FOUTIYOU TALL & 05,58 \\
CAMARA & 56513 & L'1 & M & 23 & LYCEE EL HADJ OMAR TALL & 07,17 \\
\bottomrule
\end{tabular}
\end{table}

Le tableau (table~\ref{tab:doublon}) illustre deux exemples de doublons survenus en 2012.\\
On observe que, bien que les numéros de table soient identiques, les profils associés sont différents (noms, âges, établissements, moyennes, etc.).
Ces doublons ne traduisent donc pas des répétitions, mais plutôt des erreurs de saisie ou des anomalies dans l’attribution des identifiants.

\subsection{Les valeurs manquantes dans les colonnes}

L’analyse du DataFrame révèle la présence de valeurs manquantes dans plusieurs variables. Le tableau (table~\ref{tab:valeurs_nulles}) en présente la répartition par colonne.

\begin{table}[ht]
\centering
\caption{Valeurs manquantes dans les colonnes du DataFrame des résultats du bac}
\label{tab:valeurs_nulles}
\begin{tabular}{lr}
\toprule
Colonnes & nb\_valeur\_null \\
\midrule
nom & 1 \\
prenom & 1 \\
numero\_table & 3 \\
serie & 5442 \\
sexe & 1 \\
age & 1 \\
etablissement & 1 \\
type\_candidat & 1 \\
resultat & 3977 \\
acad\_provenance & 1 \\
moy\_finale & 11 \\
mention & 945542 \\
abs & 1 \\
exclusion & 1 \\
year & 0 \\
\bottomrule
\end{tabular}
\end{table}

\subsubsection{Valeurs manquantes dans la colonne \texttt{moy\_finale}}

La colonne \texttt{moy\_finale}, qui contient la moyenne finale obtenue par les candidats, comporte 11 valeurs manquantes (table~\ref{tab:null_moyenne}).

\begin{table}[ht]
\centering
\caption{Valeurs manquantes dans la colonne moy\_finale}
\label{tab:null_moyenne}
\begin{tabular}{lrllr}
\toprule
nom & numero\_table & resultat & moy\_finale \\
\midrule
NaN & 42379 & NaN & NaN \\
XXXXXXXX & 41803 & NaN & NaN \\
XXXXXXXX & 42043 & NaN & NaN \\
XXXXXXXX & 27911 & NaN & NaN \\
XXXXXXXX & 24635 & NaN & NaN \\
XXXXXXXX & 17167 & NaN & NaN \\
XXXXXXXX & 17669 & NaN & NaN \\
XXXXXXXX & 1054 & NaN & NaN \\
NDIAYE & NaN & NaN & NaN \\
FAYE & NaN & NaN & NaN \\
KANE & NaN & NaN & NaN \\
\bottomrule
\end{tabular}
\end{table}

Ces lignes, souvent associées à des candidats absents ou exclus, ont été supprimées afin d’éviter tout biais dans l’analyse statistique et les modélisations

\subsubsection{Valeurs manquantes dans la colonne \texttt{resultat}}

La colonne \texttt{resultat} contient 3\,977 valeurs manquantes (table~\ref{tab:valeurs_nulles}). 
Après vérification croisée avec les colonnes \texttt{abs} et \texttt{exclusion}, il s’avère que ces absences de valeurs concernent exclusivement des candidats exclus ou absents. 
Par souci de cohérence, toutes les valeurs manquantes ont été remplacées par \textbf{0}, conformément à la signification officielle des résultats :

\begin{itemize}
    \item \textbf{111} : Admis d’office avec mention ;
    \item \textbf{101} : Admis après le second tour ;
    \item \textbf{100} : Passé au second tour sans succès ;
    \item \textbf{10} : Échec direct (moyenne insuffisante dès le premier tour) ;
    \item \textbf{0} : Absent, exclu ou dossier incomplet.
\end{itemize}

\subsection{Correction du type de la colonne \texttt{moy\_finale}}

Comme on peut le constater dans le code présenté (code~\ref{lst:info_bac}), le type initial de la colonne \texttt{moy\_finale} n'était pas exploitable tel quel. 
En effet, certaines valeurs contenaient des tirets, d'autres utilisaient des virgules comme séparateur décimal, et certaines étaient tout simplement vides ou non numériques.

\begin{lstlisting}[language=Python,
    caption=Correction du type de la colonne moy\_finale,
    label=lst:moy_finale_type,
    basicstyle=\ttfamily\small,
    backgroundcolor=\color{gray!10}
]
 all_data_filtre['moy_finale'] = (
    all_data_filtre['moy_finale']
    .astype(str)                         # Convertir en string
    .str.replace('-', '', regex=False)   # Supprimer les tirets
    .str.replace(',', '.', regex=False)  # les virgules en points
    .str.strip()                         # Supprimer les espaces
    .replace({'': None, 'nan': None})    # Supprimer les chaines vides
    .astype(float)                       # Convertir en float
    )
\end{lstlisting}

Pour rendre cette colonne exploitable statistiquement, plusieurs opérations de nettoyage ont été effectuées :
\begin{itemize}
\item Suppression des tirets \texttt{'-'}.
\item Remplacement des virgules \texttt{','} par des points \texttt{'.'}.
\item Suppression des espaces superflus.
\item Conversion des chaînes vides ou non valides (\texttt{''}, \texttt{'nan'}) en valeurs manquantes.
\item Conversion finale de la colonne en type \texttt{float}.
\end{itemize}

\subsection{Création des colonnes \texttt{admis} et \texttt{session}}

Dans le but de simplifier l’analyse et d’améliorer la lisibilité des résultats, deux nouvelles colonnes ont été dérivées à partir de la variable \texttt{resultat} :

\begin{itemize}
    \item \texttt{admis} : indique si un candidat a été admis ou non. Les codes \texttt{111} (admis d’office avec mention) et \texttt{101} (admis au second tour) sont considérés comme « admis », les autres comme « non admis ».
    \item \texttt{session} : précise la session de réussite du candidat. Le code \texttt{111} correspond au « 1er Tour », et le code \texttt{101} au « 2e Tour ».
\end{itemize}

\begin{lstlisting}[language=Python,
    caption=Création de nouvelles colonnes,
    label=lst:creation_colonnes,
    basicstyle=\ttfamily\small,
    backgroundcolor=\color{gray!10}
]
 # Creation de la colonne 'admis' de 'resultat'
 all_data['admis'] = all_data['resultat'].apply(lambda x: 'admis' 
                              if pd.notna(x) and x in ['111', '101'] 
                              else 'non admis'
                              )

 # Creation de la colonne 'session' de 'admis'
 all_data['session'] = all_data['resultat'].apply(lambda x: '1er Tour' 
                                if pd.notna(x) and x == '111' 
                                else ('2e Tour' 
                                      if pd.notna(x) and x == '101' 
                                      else ''
                                    ))
\end{lstlisting}

\section{Fusion des données des Inscrits et des Résultats de l’UCAD}

\subsection{Données des inscriptions à l’UCAD}

Les données d’inscription utilisées dans cette étude couvrent la période de 2002 à 2024. 
Cependant, pour garantir la cohérence avec les données de résultats (disponibles uniquement de 2011 à 2024), 
nous avons retenu uniquement les inscriptions allant de 2011 à 2024.

La sortie de code (code~\ref{lst:inscription_ucad}) présente un aperçu global de la base de données des inscriptions à l’UCAD pour les années universitaires allant de 2011 à 2023. 
Elle contient 1 141 120 enregistrements répartis sur 13 variables. 
Les variables \textit{NUMERO} et \textit{ANNEE UNIVERSITAIRE}, constituent les clés essentielles pour la fusion avec la base des résultats académiques.

\begin{lstlisting}[language=Python,
    caption=Info global du data des inscriptions, 
    label=lst:inscription_ucad, 
    basicstyle=\ttfamily\footnotesize, 
    backgroundcolor=\color{gray!10}
]
 <class 'pandas.core.frame.DataFrame'>
 RangeIndex: 1141120 entries, 0 to 1141119
 Data columns (total 14 columns):
  #   Column                 Non-Null Count    Dtype  
 ---  ------                 --------------    -----  
  0   NUMERO                 1141120 non-null  object 
  1   SEXE                   1141120 non-null  object 
  2   ANNEE_BACC             1136174 non-null  float64
  3   NATIONALITE            1141120 non-null  object 
  4   SERIE_BACC             1109114 non-null  object 
  5   ETABLISSMENT_CODE      1141120 non-null  object 
  6   NIVEAU_SECTION         1141120 non-null  object 
  7   ANNEE_INSCRIPTION      1141120 non-null  int64  
  8   ANNEE_UNIVERSITAIRE    1141120 non-null  object 
  9   TYPE_FORMATION         1141120 non-null  object 
  10  CODE_NIVEAU            1141120 non-null  int64  
  11  NIVEAU LMD ET NON LMD  1141120 non-null  object 
  12  SYSTEME                1141120 non-null  object 
  13  DEPARTEMENT FORMATION  1141120 non-null  object 
 dtypes: float64(1), int64(2), object(11)
 memory usage: 121.9+ MB
\end{lstlisting}

\subsection{Données des résultats de l’UCAD}

La sortie de code (code~\ref{lst:resultat_ucad}) présente les informations générales de la base de données contenant les résultats universitaires des étudiants de l’UCAD. 
Cette base couvre la période allant de 2011 à 2024, et concerne uniquement les établissements ayant effectué leurs délibérations sur la plateforme institutionnelle de la DISI. 
C’est pourquoi le fichier ne contient que 753 693 enregistrements, un total largement inférieur aux 1 141 120 inscrits.

\begin{lstlisting}[language=Python,
    caption=Info global du data des résultats, 
    label=lst:resultat_ucad, 
    basicstyle=\ttfamily\footnotesize, 
    backgroundcolor=\color{gray!10}
]
 <class 'pandas.core.frame.DataFrame'>
 RangeIndex: 753693 entries, 0 to 753692
 Data columns (total 3 columns):
  #   Column               Non-Null Count   Dtype 
 ---  ------               --------------   ----- 
  0   NUMERO               753693 non-null  object
  1   ANNEE UNIVERSITAIRE  753693 non-null  object
  2   RESULTAT             753562 non-null  object
 dtypes: object(3)
 memory usage: 17.3+ MB
\end{lstlisting}

\subsection{Fusion des données d’inscription et de résultats}

Pour combiner les données d’inscription et les résultats académiques des étudiants de l’UCAD,
 une jointure a été effectuée en utilisant les colonnes \texttt{NUMERO} (identifiant étudiant) et \texttt{ANNEE UNIVERSITAIRE} comme clés.

\begin{lstlisting}[language=Python,
    caption=Jointure des données d’inscription et de résultats,
    label=lst:jointure_ucad,
    basicstyle=\ttfamily\small,
    backgroundcolor=\color{gray!10}
]
 df_final = pd.merge(df_inscrit,                 # Premier DataFrame
           df_resultat,                          # Deuxieme DataFrame
           on=['NUMERO', 'ANNEE UNIVERSITAIRE'], # Cles de jointure
           how='left'                            # le type de jointure
            ) 
\end{lstlisting}

La jointure a été réalisée à l’aide de l’option \texttt{left} pour garantir la conservation de l’ensemble des inscriptions, 
y compris celles des étudiants dont les résultats n’ont pas été délibérés ou publiés sur la plateforme académique de la DISI

\begin{lstlisting}[language=Python,
    caption=Info global du data des inscriptions et résultats, 
    label=lst:inscription_resultat_ucad, 
    basicstyle=\ttfamily\footnotesize, 
    backgroundcolor=\color{gray!10}
]
 <class 'pandas.core.frame.DataFrame'>
 RangeIndex: 1215837 entries, 0 to 1215836
 Data columns (total 15 columns):
  #   Column                 Non-Null Count    Dtype  
 ---  ------                 --------------    -----  
  0   NUMERO                 1215837 non-null  object 
  1   SEXE                   1215837 non-null  object 
  2   ANNEE_BACC             1210891 non-null  float64
  3   NATIONALITE            1215837 non-null  object 
  4   SERIE_BACC             1183525 non-null  object 
  5   ETABLISSMENT_CODE      1215837 non-null  object 
  6   NIVEAU_SECTION         1215837 non-null  object 
  7   ANNEE_INSCRIPTION      1215837 non-null  int64  
  8   ANNEE UNIVERSITAIRE    1215837 non-null  object 
  9   TYPE_FORMATION         1215837 non-null  object 
  10  CODE_NIVEAU            1215837 non-null  int64  
  11  NIVEAU LMD ET NON LMD  1215837 non-null  object 
  12  SYSTEME                1215837 non-null  object 
  13  DEPARTEMENT FORMATION  1215837 non-null  object 
  14  RESULTAT               750198 non-null   object 
 dtypes: float64(1), int64(2), object(12)
 memory usage: 139.1+ MB
\end{lstlisting}

On observe que la colonne \texttt{RESULTAT} contient un nombre important de valeurs manquantes qui s’explique par le fait que certains établissements ne délibéresnt via la plateforme de la DISI.
C'est pourquoi, ils avaient moins de lignes dans la même periode dans les données des resultats.

\section{Filtage des données}

Afin de concentrer l’analyse sur la population cible et les séries directement concernées par les réformes du baccalauréat, une étape de filtrage rigoureuse a été appliquée aux bases de données. 
L’objectif est d’exclure les individus et les séries ne correspondant pas au périmètre de l’étude.

\subsection{Nationalité}

La variable \texttt{NATIONALITE} a été filtrée pour ne conserver que les étudiants de nationalité sénégalaise. 
Ce choix permet de se concentrer sur les élèves issus du système éducatif sénégalais, et d’évaluer plus précisément l’effet des réformes nationales sur leurs trajectoires.

\subsection{Séries du baccalauréat}

L’étude porte principalement sur les séries impactées par les réformes éducatives récentes, notamment l’introduction du baccalauréat arabe (2013) et de la série STEG (2019). Ainsi, seules les séries suivantes ont été retenues :

\begin{itemize}
    \item \textbf{Séries arabes} : \texttt{L1-AR}, \texttt{S1-AR}, \texttt{S2-AR} ;
    \item \textbf{Séries franco-arabes} : \texttt{LA}, \texttt{S1A}, \texttt{S2A} ;
    \item \textbf{Série STEG} : \texttt{STEG} ;
    \item \textbf{Série G} : \texttt{G} (anciennement série G, remplacée progressivement par la STEG) ;
    \item \textbf{Séries de référence (littéraire et scientifique)} :
        \begin{itemize}
            \item \texttt{L'1} : utilisée comme référence pour les séries littéraires non arabes ;
            \item \texttt{S1} et \texttt{S2} : références pour les filières scientifiques.
        \end{itemize}
\end{itemize}

Ce choix permet de comparer les performances et parcours entre les anciennes séries, les nouvelles séries réformées, et les séries de référence.

\subsection{Niveau universitaire}

L’analyse se limite aux étudiants inscrits dans le \textbf{premier cycle universitaire}, c’est-à-dire les niveaux \texttt{L1}, \texttt{L2} et \texttt{L3}. 
Ce choix vise à observer les effets immédiats et à court terme des réformes du baccalauréat sur l’insertion et la progression dans les premières années à l’université.

\subsection{DataFrame final}

Après l’ensemble des étapes de filtrage, de nettoyage et de fusion, le \texttt{DataFrame} final contient un total de 579\,039 enregistrements correspondant à des étudiants de nationalité sénégalaise, inscrits à l’UCAD entre 2011 et 2024, issus des séries ciblées, et se situant dans le premier cycle universitaire.

\begin{lstlisting}[language=Python,
    caption=Filtrage des données,
    label=lst:filtrage,
    basicstyle=\ttfamily\small,
    backgroundcolor=\color{gray!10}
]
 <class 'pandas.core.frame.DataFrame'>
 Index: 579039 entries, 7 to 1215834
 Data columns (total 15 columns):
  #   Column                 Non-Null Count   Dtype  
 ---  ------                 --------------   -----  
  0   NUMERO                 579039 non-null  object 
  1   SEXE                   579039 non-null  object 
  2   ANNEE_BACC             579039 non-null  float64
  3   NATIONALITE            579039 non-null  object 
  4   SERIE_BACC             579039 non-null  object 
  5   ETABLISSMENT_CODE      579039 non-null  object 
  6   NIVEAU_SECTION         579039 non-null  object 
  7   ANNEE_INSCRIPTION      579039 non-null  int64  
  8   ANNEE UNIVERSITAIRE    579039 non-null  object 
  9   TYPE_FORMATION         579039 non-null  object 
  10  CODE_NIVEAU            579039 non-null  int64  
  11  NIVEAU LMD ET NON LMD  579039 non-null  object 
  12  SYSTEME                579039 non-null  object 
  13  DEPARTEMENT FORMATION  579039 non-null  object 
  14  RESULTAT               400944 non-null  object 
 dtypes: float64(1), int64(2), object(12)
 memory usage: 70.7+ MB
\end{lstlisting}

\section{Conclusion}

Ce chapitre a permis de mettre en place une base de données propre, cohérente et ciblée, indispensable pour mener les analyses ultérieures.

Le \texttt{DataFrame} final obtenu constitue le socle de notre étude. Il résulte de la liaison entre les données d’inscription à l’UCAD, qui contiennent les informations sur les séries et années du baccalauréat des étudiants, et les données de résultats universitaires.
Cette intégration permet de relier les parcours académiques aux caractéristiques du bac.