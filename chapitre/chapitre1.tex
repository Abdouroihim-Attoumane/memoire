\chapter{État de l’art}

\section{Introduction}

Depuis sa création, comme énoncé précédemment, le baccalauréat sénégalais a connu de nombreuses évolutions, tant dans sa structure que dans ses objectifs. 
Initialement conçu comme un simple diplôme de fin d’études secondaires, il est aujourd’hui devenu un véritable levier d’accès à l’enseignement supérieur et un indicateur clé de performance du système éducatif national.

Ce chapitre propose une revue synthétique des principales réformes ayant marqué l’histoire du baccalauréat au Sénégal, notamment celles de 1995, 2000, 2013 (introduction du bac arabe) et 2019 (création de la série STEG). 
Il s’agira d’examiner le contexte de leur mise en œuvre, les motivations qui les ont justifiées, ainsi que leurs ambitions en termes de modernisation, d’inclusion et de diversification des profils d’élèves.

Cette mise en perspective historique permet de mieux situer l’organisation de notre étude, qui vise à évaluer l’impact réel de ces réformes sur les performances scolaires (taux de réussite) et les parcours universitaires (suivi de cohortes à l’UCAD). 
Elle constitue ainsi une étape essentielle pour comprendre les enjeux de notre analyse et justifier le recours à une approche fondée sur les données et la visualisation interactive. 

\section{Historique et rôle du baccalauréat}

Le baccalauréat a été inventé en France au XIX\textsuperscript{e} siècle par un décret de l'empereur Napoléon Ier en 1808.
À l’origine appelé le \textbf{«~Bachot~»}, il tire son étymologie du latin médiéval \textit{ "bacca laurea"}, il désigne en latin médiéval \textbf{la couronne de laurier} remise aux vainqueurs \cite{bacHistorique}.

Dans le contexte sénégalais, il a été progressivement intégré au système éducatif colonial, puis nationalisé après l’indépendance. 
Il sanctionne la fin du cycle secondaire et donne accès à l’université, devenant ainsi un indicateur majeur de performance du système éducatif.

Aujourd’hui, le baccalauréat incarne à la fois une étape symbolique vers l’âge adulte et un enjeu stratégique pour le développement du capital humain, d’où l’importance des réformes visant à le rendre plus inclusif et mieux adapté aux réalités sociales et économiques du pays.

\section{Organisation du baccalauréat : structure et acteurs}

Le baccalauréat Sénégalais est organisé par \textbf{l'Office du baccalauréat}, rattaché à l'Université Cheikh Anta Diop de Dakar (UCAD), ce qui renforce l'idée qu'il constitue le premier diplôme universitaire.
Chaque année, son organisation mobilise plusieurs milliers d’acteurs : enseignants, surveillants, correcteurs et présidents de jury.

L’examen est structuré autour de différentes séries, réparties en trois grandes filières : \textbf{littéraire}, \textbf{scientifique} et \textbf{tertiaire}. 
Cette diversification vise à mieux prendre en compte la pluralité des profils et des parcours de formation des élèves. Les épreuves s’appuient sur des programmes nationaux fixés par le ministère de l'Éducation, et sont organisées de manière centralisée, tant dans leur déroulement que dans leur correction.

Ce travail est réalisé en étroite collaboration avec l’Office du Baccalauréat, ce qui permet un accès privilégié à certaines données et une meilleure compréhension des mécanismes internes liés à l’organisation et à l’évolution de l’examen.

\section{Réformes du baccalauréat au Sénégal}

Depuis les années 1990, plusieurs réformes ont été introduites pour adapter le baccalauréat aux réalités sociolinguistiques, 
économiques et pédagogiques du pays.

\subsection{Réformes de 1995 (décret n° 95-947)}

Certainement la réforme la plus marquante, elle s'inscrit dans le cadre de la concertation nationale sur l'enseignement supérieur tenu le 9 décembre 1995, 
dont sont issues les propositions de la commission qui avait été chargée de réfléchir sur l'examen du baccalauréat \cite{decret95}.

Elle a introduit les changements majeurs dans la structure des séries telles qu'on les connaît aujourd'hui, 
notamment par l'article 7 du décret établissant les choix des séries que les candidats devront choisir au moment de leurs inscriptions :

\begin{itemize}
    \item Série L1 : Langues et Civilisations
    \item Série L2 : Sciences sociales et humaines
    \item Série G : Techniques quantitatives d'économie et de gestion
    \item Série S1 : Sciences exactes (Mathématiques et Physique)
    \item Série S2 : Sciences expérimentales
    \item Série S3 : Sciences et Techniques
    \item Série T1 : Fabrication mécanique
    \item Série T2 : Électrotechnique-Électronique
\end{itemize}

C'est pour modifier et compléter cette article 7 du décret n°1995-947 que toutes les réformes étudiées dans ce mémoire ont été introduites.

\subsection{Réformes de 2000 (décret n° 2000-585)}

Apparue très tôt dans l'élémentaire, la langue arabe sera reconnue comme langue vivante étrangère au Sénégal, laissée au choix de l'élève dans l'enseignement moyen, puis secondaire.

Parallèlement à cette évolution dans l'enseignement public, l'initiative privée, portée par une demande socio-culturelle, a donné naissance à un système d'enseignement arabe encore embryonnaire.
Ainsi, avec la multiplication des établissements exclusivement arabe ou bilingue franco-arabe, la nécessité d'encadrer ce phénomène a conduit l'État sénégalais à mettre en place un référentiel de diplôme : 
c'est la création du \textbf{Certificat arabe} et du \textbf{BFEM arabe} \cite{decret2000}.

La création du collège public franco-arabe \textbf{Mouhamadou Fadilou Mbacké} de Dakar en 1963 procéde de cette volonté de répondre à cette demande sociale. Un second cycle y a été mis en place, offrant les même perspectives que celles proposées aux autres collégiens.
Ainsi, le décret n°2000-585 du 6 juillet 2000 a introduit le baccalauréat option arabe qui modifie l'article 7 du décret n° 95-947 en un article 7 bis en intégrant les baccalauréats option arabe suivants : 

\begin{itemize}
    \item Langues et Sciences sociales : LA
    \item Sciences fondamentales : S1A
    \item Sciences appliquées : S2A
\end{itemize}

\subsection{Réformes de 2013 (décret n° 2013-913)}

\textit{"Le baccalauréat organisé (matérialisé) par le décret n°2000-586 du 20 juillet 2000, 
n'a pas atteint tous ses objectifs car des écoles privées franco-arabes et des instituts islamiques se multiplient entraînant un manque de contrôle
sur les programmes enseignés ainsi qu'une prolifération de diplômes qui empêche l'existence d'un standard commun à toutes les études secondaires du pays"}\cite{decret2013}.

C'est pour encadrer ce phénomène que, tout comme les études secondaires franco-arabes, les études secondaires arabes seront sanctionnées par les baccalauréats arabes. 
Ces derniers sont définis dans un article 7 ter, qui complète l'article 7 bis du décret n° 2000-585, en intégrant les séries suivantes :

\begin{itemize}
    \item Littératures et Civilisations arabes : L-AR 
    \item Mathématiques et Sciences Physique : S1-AR
    \item Sciences expérimentales : S2-AR
\end{itemize} 

\textbf{Remarque importante :}
Il convient de souligner que, bien que le décret n° 2013-913 ait introduit les séries \textbf{S1-AR} et \textbf{S2-AR},
les données disponibles et la pratique sur le terrain indiquent que ces séries scientifiques n'ont \textbf{jamais été concrètement mises en place ni organisées} depuis la promulgation de ce décret.
Seule la série L-AR est actuellement opérationnelle.

\subsection{Réformes de 2019(décret n° 2019-645)}

À la difference des précédentes réformes, celle-ci à pour but de renforcer la série G aux exigences des Formations Professionnelles et Techniques(FPT)
et des Instituts Supérieurs d'Enseignement Professionnel (ISEP).

\textit{"C'est dans ce cadre qu'il est proposé que la série G (économie et gestion) soit transformée en série technologique (sciences et technologique de l'économie et de la gestion : STEG)
visant essentiellement à installer chez les élèves les compétences en associant la culture générale et la technologie"}\cite{decret2019}.

\section{Travaux antérieures }

Plusieurs études ont porté sur la performance du système éducatif sénégalais et l’évolution du baccalauréat.

L’article de Diagne (2023) analyse l’évolution du taux de réussite au bac sur deux décennies (2001–2022). 
Il met en évidence l’instabilité chronique du système, marquée par une réussite oscillant autour de 40~\%, bien inférieure aux standards internationaux. 
L’auteur identifie plusieurs facteurs influents : interruptions pédagogiques, surcharge des programmes, conditions socio-économiques, etc. 
Malgré les nombreuses réformes (EPT 1990, PDEF 2003, PAQUET-ET 2013), les performances restent en deçà des attentes \cite{Mbaye2023}.

Par ailleurs, le rapport de la DES de l’UCAD (2021) adopte une approche longitudinale de suivi de cohorte pour évaluer l’efficacité interne des formations universitaires. 
À partir de la cohorte 2013-2014, il retrace les parcours des étudiants selon leur série de bac et leur établissement d’accueil à l’UCAD. 
Cette étude montre des taux importants de redoublement et d’abandon dès les premières années universitaires, avec des écarts marqués selon les séries d’origine \cite{des2021}.

Ces travaux révèlent un manque d’articulation entre les réformes du bac et les trajectoires universitaires, d’où l’importance d’une évaluation intégrée comme celle proposée ici.

\section{Approche méthodologique et innovante de l'étude}

Ce travail s’inscrit dans le cadre d’un stage à la Direction des Études et des Statistiques (DES) de l’UCAD, en collaboration avec l’Office du Baccalauréat. 
Il combine une analyse data-driven et des méthodes de machine learning pour évaluer l’impact des réformes sur deux axes :
\begin{enumerate}
    \item \textbf{performance académique}: Analyse de l'évolution du taux de réussite au bac de 2006 à 2024 globale et des série concerner, via des modèles de régression.
    \item \textbf{performance universitaire}: Une suivit des cohortes ici des séries concerner du bac, avec une attention particulière à leur insertion et leur progression à l’UCAD. 
\end{enumerate}
Un \textbf{tableau de bord interactif} sera mis en place pour visualiser les résultats de l'analyse et permettre une exploration dynamique des données, en vue de faciliter la prise de décision et la formulation de recommandations éclairées.

\section{Conclusion}

Ce premier chapitre a permis de retracer l’évolution historique et structurelle du baccalauréat au Sénégal, en mettant en évidence les principales réformes qui l’ont façonné depuis les années 1990. 
Motivées par des enjeux pédagogiques, socioculturels et économiques, ces réformes ont conduit à une transformation en profondeur du système : modification des séries existantes, introduction de nouvelles séries, et adaptation progressive aux réalités nationales.

Elles témoignent de la volonté des autorités de moderniser et de diversifier l’offre éducative afin de mieux répondre aux besoins de la société sénégalaise. 
Cette contextualisation est indispensable pour comprendre les fondements de notre étude, qui vise à évaluer l’impact concret de ces réformes à travers une approche analytique et prédictive, fondée sur l’exploitation rigoureuse de données réelles.
