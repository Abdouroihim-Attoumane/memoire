\chapter{État de l’art}

\section{Introduction}

Depuis sa création, comme énoncé pécedemment, le baccalauréat a connu de nombreuses évolutions, tant dans sa structure que dans sa finalité.
Ce chapitre propose une revue des principales réformes ayant marqué le système du baccalauréat au Sénégal, en mettant en lumière leur contexte, leur portée et leurs objectifs.
Il vise également à situer l'organisation de notre étude dans ce cadre évolutif. 

\section{Historique et rôle du baccalauréat}

Le baccalauréat a été inventé en France au XIXe siécle par un décret de l'empereur Napoléon Ier en 1808.
Au commencement le baccalauréat s'appelait le \textbf{Bachot}.
Son étymologie \textit{ "bacca laurea"} désigne en latin médiéval, \textbf{la couronne de laurier} remise aux vainqueurs \cite{bacHistorique}.

Dans le contexte sénégalais, il a été progressivement intégré au système éducatif colonial, puis nationalisé après l’indépendance. 
Il sanctionne la fin du cycle secondaire et donne accès à l’université, devenant ainsi un indicateur majeur de performance du système éducatif.

\section{Organisation du baccalauréat : structure et acteurs}

Le baccalauréat Sénégalais est organisé par \textbf{l'Office du baccalauréat}, rattaché à l'Université Cheikh Anta Diop de Dakar (UCAD), ce qui renforce l'idée qu'il constitue le premier diplôme universitaire.
Il mobilise chaque année plusieurs milliers d’enseignants, surveillants, correcteurs et examinateurs.

Il se décline en différentes séries réparties en trois filières distinctes : \textbf{littéraire}, \textbf{scientifique} et \textbf{tertiaire}, afin de répondre à la diversité des profils et des parcours de formation. 

L’examen repose sur des programmes définis par le ministère en charge de l’éducation nationale, avec une centralisation de l’organisation et des corrections.

\section{Réformes du baccalauréat au Sénégal}

Depuis les années 1990, plusieurs réformes ont été introduites pour adapter le baccalauréat aux réalités sociolinguistiques, 
économiques et pédagogiques du pays.

\subsection{Réformes de 1995 (décret n° 95-947)}

Certainement la réforme la plus marquante, elle s'inscrit dans le cadre de la concertation nationale sur l'enseignement supérieur tenu le 9 décembre 1995, 
dont sont issues les propositions de la commission qui avait été chargée de réfléchir sur l'examen du baccalauréat \cite{decret95}.

Elle a introduit les changements majeurs dans la structure des séries telles qu'on les connait aujourd'hui, 
notamment par l'article 7 du décret établissant les choix des séries que les candidats devront choisir au moment de leurs inscriptions :

\begin{itemize}
    \item Série L1 : Langues et Civilisations
    \item Série L2 : Sciences sociales et humaines
    \item Série G : Techniques quantitatives d'économie et de gestion
    \item Série S1 : Sciences exactes (Mathématiques et Physique)
    \item Série S2 : Sciences expérimentales
    \item Série S3 : Sciences et Techniques
    \item Série T1 : Fabrication mécanique
    \item Série T2 : Electrotechnique-Electronique
\end{itemize}

C'est pour modifier et compléter cette article 7 du décret n°1995-947 que toutes les réformes étudiées dans ce mémoire ont été introduites.

\subsection{Réformes de 2000 (décret n° 2000-585)}

Apparue très tôt dans l'élémentaire, la langue arabe sera reconnue comme langue vivante étrangère au Sénégal, laissée au choix de l'élève dans l'enseignement moyen, puis secondaire.

Parallélement à cette évolution dans l'enseignement public, l'initiative privée, portée par une demande socio-culturelle, a donné naissance à un système d'enseignement arabe encore embryonnaire.
Ainsi, avec la multpilication des établissements exclusivement arabe ou bilingue franco-arabe, la nécessité d'encadrer ce phénomène a conduit l'État sénégalais à mettre en place un référentiel de diplôme : 
c'est la création du \textbf{Certificat arabe} et du \textbf{BFEM arabe} \cite{decret2000}.

La création du collège public franco-arabe \textbf{Mouhamadou Fadilou Mbacké} de Dakar en 1963 procéde de cette volonté de répondre à cette demande sociale. Un second cycle y a été mis en place, offerant les même perspectives que celles proposées aux autres collègiens.
Ainsi, le décret n°2000-585 du 6 juillet 2000 a introduit le baccalauréat option arabe qui modifie l'article 7 du décret n° 95-947 en un article 7 bis en intégrant les baccalauréats option arabe suivants : 

\begin{itemize}
    \item Langues et Sciences sociales : LA
    \item Sciences fondamentales : S1A
    \item Sciences appliquées : S2A
\end{itemize}

\subsection{Réformes de 2013 (décret n° 2013-913)}

\textit{"Le baccalauréat organisé (matérialisé) par le décret n°2000-586 du 20 juillet 2000, 
n'a pas atteint tous ses objectifs car des écoles privées franco-arabes et des instituts islamiques se multiplient entraînant un manque de contrôle
sur les programmes enseignés aini qu'une prolifération de diplômes qui empêche l'existence d'un standard commun à toutes les études secondaires du pays"}\cite{decret2013}.

C'est pour encadrer ce phénomène que, tout comme les études secondaires franco-arabes, les études secondaires arabes seront sanctionnées par les baccalauréats arabes. 
Ces derniers sont définis dans un article 7 ter, qui complète l'article 7 bis du décret n° 2000-585, en intégrant les séries suivantes :

\begin{itemize}
    \item Littératures et Civilisations arabes : L-AR 
    \item Mathématiques et Sciences Physique : S1-AR
    \item Sciences expérimentales : S2-AR
\end{itemize} 

\subsection{Réformes de 2019(décret n° 2019-645)}

À la difference des précédentes réformes, celle-ci à pour but de renforcer la série G aux exigences des Formations Professionnelles et Techniques(FPT)
et des Instituts Supérieurs d'Enseignement Professionnel (ISEP).

\textit{"C'est dans ce cadre qu'il est proposé que la série G (éconmie et gestion) soit transformée en série technologique (sciences et technologique de l'économie et de la gestion : STEG)
visant essentiellement à installer chez les élèves les compétences en associant la culture générale et la technologie"}\cite{decret2019}.

\section{Travaux antérieures }

\section{Approche méthodologique et innovante de l'étude}

Ce travail s’inscrit dans le cadre d’un stage à la Direction des Études et des Statistiques (DES) de l’UCAD, en partenariat avec l’Office du Baccalauréat. 
Il combine une analyse data-driven et des méthodes de machine learning pour évaluer l’impact des réformes sur deux axes :
\begin{enumerate}
    \item \textbf{performance académique}: Analyse de l'évolution du taux de réussite au bac de 2006 à 2024 globale et des série concerner, via des modèles de régression.
    \item \textbf{performance universitaire}: Une suivit des cohortes ici des séries concerner du bac 
\end{enumerate}
Un \textbf{tableau de bord interactif} sera mis en place pour visualiser les résultats de l'analyse et permettre une exploration dynamique des données.

\section{Conclusion}

Ce premier chapitre a permis de retracer l’évolution historique et structurelle du baccalauréat au Sénégal, en mettant en lumière les principales réformes qui l’ont façonné depuis les années 1990. 
Ces réformes, souvent motivées par des enjeux pédagogiques, socioculturels ou économiques, ont profondément modifié les séries existantes, introduit de nouvelles filières, et adapté l’examen aux réalités locales. 
Elles traduisent la volonté constante des autorités de moderniser et de diversifier le système éducatif pour mieux répondre aux besoins de la société sénégalaise. Cette contextualisation est essentielle pour comprendre la portée de notre étude, 
qui se propose d’évaluer l’impact de ces réformes à travers une approche analytique et prédictive, fondée sur l’exploitation des données réelle.