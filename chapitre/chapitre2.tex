\chapter{Données et outils utilisés}

\section{Introduction}

Ce chapitre introduit les données exploitées ainsi que les outils utilisés pour mener à bien notre analyse. 
Il constitue une étape essentielle dans la construction de l’étude, car la qualité et la structuration des données conditionnent la pertinence des résultats.

Les données utilisées couvrent une période suffisamment large pour permettre des comparaisons temporelles et une évaluation rigoureuse des effets des réformes. 
Elles ont été préparées et organisées en vue de faciliter les analyses statistiques ainsi que les visualisations.

Les outils mobilisés permettent à la fois le traitement, l’analyse et la visualisation claire et dynamique des résultats. 

\section{Sources et Description des données}

\subsection{Sources de données}

Ce projet s’appuie sur trois principales sources de données fournies par deux institutions officielles : 
\begin{itemize}
    \item \textbf{l’Office du Baccalauréat du Sénégal}, chargé de l’organisation de l’examen et de la production des résultats officiels du bac sur l’ensemble du territoire national ;
    \item \textbf{la Direction de l’Informatique et des Systèmes d’Information (DISI)} de l’UCAD, responsable de la gestion des données académiques et administratives à l’université ;
\end{itemize}

\subsection{Description des données}


L’étude repose sur trois ensembles de données complémentaires, couvrant le parcours des apprenants depuis l’obtention du baccalauréat jusqu’à leur progression dans l’enseignement supérieur à l’UCAD.

\subsubsection{Résultats du Baccalauréat (2006--2024)}

Ces données ont été fournies par l’Office du Baccalauréat du Sénégal. 
Elles contiennent les informations relatives aux candidats (année, série, résultat, mention, session, etc.). 
Elles permettent de mesurer les taux de réussite globaux et par série, et de suivre l’impact des réformes.

\subsubsection{Inscriptions universitaires à l’UCAD (2002--2024)}

Issues de la Direction de la DISI, ces données renseignent sur le profil des étudiants inscrits à l’UCAD : année d’inscription, série d’origine au bac, établissement d’accueil, etc. 
Elles offrent une vision globale de l’évolution des flux d’entrée à l’université selon les caractéristiques des bacheliers.

\subsubsection{Résultats académiques à l’UCAD (2010--2024)}

Ces données, également fournies par la DISI, détaillent la performance universitaire des étudiants : moyennes annuelles, crédits, session, mention, résultats. 
Elles permettent d’apprécier la réussite dans le supérieur en fonction du parcours scolaire initial.

\bigskip

La combinaison de ces trois bases~--~résultats au bac, inscriptions et performances universitaires~--~permet de reconstruire des trajectoires individuelles et de réaliser un suivi de cohorte. 
Grâce aux variables communes (année, série du bac, résultat, etc.), il est possible de lier le profil d’entrée des étudiants à leur évolution à l’université, et d’évaluer ainsi l’impact des séries du baccalauréat sur leur réussite postscolaire.

\section{Outils et Technologies utilisés}

\subsection{Python et ses bibliothèques }

\includegraphics[width=3cm]{images/python.png}

Python est un langage de programmation open source, interprété, 
simple à apprendre et largement utilisé dans le domaine scientifique et technique.
Python s’est imposé comme l’un des outils les plus puissants pour l’analyse de données et le développement de modèles d’intelligence artificielle, 
notamment en raison de la richesse de ses bibliothèques spécialisées \cite{python}.

Dans le cadre de cette étude, Python a été utilisé à la fois pour le traitement, l’exploration et la visualisation des données.

\subsubsection{bibliothèques utilisées}

\includegraphics[width=3cm]{images/Pandas_logo.png}

Pandas est une bibliothèque Python spécialisée dans la manipulation et l’analyse de données. 
Elle offre des structures de données flexibles et efficaces, notamment les \textbf{DataFrame}, 
qui permettent de gérer facilement des tableaux de données similaires à ceux d'Excel ou d'une base de données\cite{pandas}.

C’est sans doute le package le plus utilisé en science des données. Grâce à ses nombreuses fonctionnalités,
il ma permet de charger, filtrer, nettoyer et transformer les jeux de données de manière rapide et intuitive.

\includegraphics[width=4cm]{images/Matplotlib.png}

Matplotlib est une bibliothèque Python dédiée à la visualisation de données. 
Elle permet de créer une grande variété de graphiques statiques. 
Son interface simple et sa compatibilité avec les structures de données 
comme les DataFrame de pandas en font un outil de choix pour représenter visuellement les résultats d’une analyse \cite{matplotlib}.

\newpage
\includegraphics[width=4cm]{images/seaborn.png}

Seaborn est une bibliothèque Python construite sur Matplotlib et conçue pour simplifier la visualisation statistique des données. 
Elle offre des graphiques esthétiques et informatifs avec moins de code, tout en étant parfaitement compatible avec les structures de données comme les DataFrame de pandas \cite{seaborn}.


% \includegraphics[width=3cm]{images/sckitlearn.png}

% Scikit-learn est une bibliothèque Python dédiée au machine learning. 
% Elle offre des outils simples et efficaces pour appliquer des modèles de classification, de régression et de clustering \cite{scikitLearn}.

% Dans mon projet, je l’ai utilisée pour prédire le taux de réussite au baccalauréat. 
% Elle m’a aussi permis d’évaluer les performances des modèles grâce à des métriques comme (RMSE).

% \includegraphics[width=5cm]{images/statsmodes.png}

% Statsmodels est une bibliothèque Python conçue pour l’estimation de modèles statistiques. 
% Elle est particulièrement utilisée pour les analyses de régression, les séries temporelles et les tests statistiques \cite{statsmodels}.

% Dans mon projet, Statsmodels m’a servi à réaliser des régressions linéaires et à effectuer des tests d’hypothèses. 
% Elle m’a permis d’interpréter les relations entre les variables, grâce à des résultats détaillés incluant les coefficients, 
% les p-values et les intervalles de confiance.

\subsection{Power BI}

\includegraphics[width=4cm]{images/powerbi.png}

Power BI est un outil de visualisation et d’analyse de données développé par Microsoft. 
Il permet de créer des tableaux de bord interactifs et dynamiques à partir de diverses sources de données, 
facilitant ainsi l'exploration visuelle et la prise de décision basée sur les données \cite{powerBI}.

Dans le cadre de mon projet, j’ai utilisé Power BI pour concevoir un tableau de bord interactif regroupant les statistiques du baccalauréat de 2006 à 2024. 
Ce tableau de bord, présenté dans le chapitre 6, est destiné à l’Office du Baccalauréat. Il a pour objectif de faciliter l’analyse des données historiques du bac et de servir de support à toute nouvelle étude portant sur l’évolution du système éducatif.

\section{Conclusion}

Ce chapitre a permis de présenter les données exploitées ainsi que les outils mobilisés pour leur traitement, leur analyse et leur visualisation. 
Les trois jeux de données, provenant de sources officielles telles que l’Office du Baccalauréat et la DISI de l’UCAD, 
offrent une base solide pour évaluer l’impact des réformes éducatives sur les résultats au bac et l’insertion universitaire. 
Par ailleurs, l’utilisation d’outils puissants comme Python et ses bibliothèques spécialisées, ainsi que Power BI pour la visualisation, 
garantit une analyse rigoureuse, reproductible et accessible. Ces ressources forment ainsi le socle méthodologique sur lequel reposent les analyses menées dans les chapitres suivants.