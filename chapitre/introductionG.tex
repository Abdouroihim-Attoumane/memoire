\chapter*{Introduction Générale}
\markboth{\large Introduction Générale}{}
\addcontentsline{toc}{chapter}{Introduction Générale}

\section*{Contexte et justification}
\addcontentsline{toc}{section}{Contexte et justification}

Le système éducatif sénégalais, hérité du modèle français, repose sur une structuration en trois cycles : primaire, moyen et secondaire. 
La fin du secondaire est sanctionnée par le baccalauréat, diplôme pivot marquant l'accès à l'enseignement supérieur et considéré comme le premier grade universitaire. 
Il joue un rôle central, non seulement en tant que certificat de fin d’études, mais aussi comme indicateur de performance du système éducatif.

Malgré les réformes successives visant à adapter ce diplôme aux réalités nationales et aux enjeux contemporains, le baccalauréat au Sénégal reste confronté à un taux de réussite relativement faible, 
oscillant autour de 40\%, loin des standards observés dans des pays comme la France ou le Canada où il s'oscille au tour de 99\% \cite{Mbaye2023}. 
Ce paradoxe entre le nombre croissant de candidats et la stagnation du taux de réussite interroge sur l’efficacité des politiques éducatives mises en œuvre.

Dans ce contexte, plusieurs réformes majeures ont été introduites : l’instauration du baccalauréat arabo-français en 2000, la création du bac arabe en 2013, 
et la transformation de la série G en série STEG (Sciences et Techniques de la Gestion) en 2019. 
Ces réformes traduisent la volonté des autorités d’élargir les opportunités d’accès à l’enseignement supérieur, de diversifier les profils de bacheliers, 
et d'améliorer l'adéquation entre formation et marché du travail. Ce mémoire se propose d’évaluer l’impact réel de ces transformations sur le taux de réussite au baccalauréat et sur l’insertion universitaire des bacheliers, notamment à l'UCAD.

\section*{Problématique et Objectifs de l'étude}
\addcontentsline{toc}{section}{Problématique et Objectifs de l'étude}

Les réformes éducatives ont-elles permis de redynamiser le baccalauréat sénégalais et d’améliorer l’accès à l’enseignement supérieur ? 
Malgré l’intention affichée de moderniser le système et de le rendre plus inclusif, plusieurs interrogations subsistent :
\begin{itemize}
    \item Ces réformes ont-elles eu un impact significatif sur les taux de réussite ?
    \item Comment les nouvelles séries, comme le bac arabe ou la série STEG, se positionnent-elles en termes de performances au bac ?
    \item Comment les bacheliers issus des séries arabes évoluent-ils dans l’enseignement supérieur, notamment à l’UCAD ?
    \item Existe-t-il des disparités significatives entre les différentes séries du baccalauréat en termes de réussite et d’insertion à l'UCAD ?
\end{itemize}

C'est à l'analyse de ces questions que s'attelle cette étude.
Ce travail vise ainsi à combler un manque crucial d’évaluation quantitative des réformes éducatives, 
tout en proposant des outils analytiques pour guider les futures politiques publiques. 
En croisant l’exploitation de données massives (résultats du bac, inscriptions à l’UCAD) et des méthodes avancées de data science, 
il offre une vision factuelle pour optimiser l’efficacité du système éducatif sénégalais.

\section*{Méthodologie et Organisation du document}
\addcontentsline{toc}{section}{Méthodologie et Organisation du document}

%Ce mémoire est structuré comme suit :
\begin{enumerate}
    \item Le premier chapitre présente l'état de l’art sur les réformes du baccalauréat au Sénégal,...
    \item Le deuxième chapitre présente les données et outils utilisés, détaillant les outils statistiques et analytiques utilisés pour évaluer l'impact des réformes.
    \item Le troisième chapitre traite l'exploration et la préparation des données,...
    \item Le quatrième chapitre présente l'analyse statistique des impacts des réformes,...
    \item Le cinquième chapitre présente le Dashboard Power Bi de visualisation des résultats au baccalauréat de 2006 à 2024 à fin de facilité toute nouvelle étude.
    \item Enfin, mes recommandations et conclusions sont présentées, soulignant les implications de cette étude pour les politiques éducatives futures.
\end{enumerate}