\chapter*{Introduction Générale}
\markboth{\large Introduction Générale}{}
\addcontentsline{toc}{chapter}{Introduction Générale}

\section*{Contexte et justification}
\addcontentsline{toc}{section}{Contexte et justification}

Le système éducatif sénégalais, hérité du modèle français, repose sur une structuration en trois cycles : primaire, moyen et secondaire. 
La fin du secondaire est sanctionnée par le baccalauréat, diplôme pivot marquant l'accès à l'enseignement supérieur et considéré comme le premier grade universitaire. 
Il joue un rôle central, non seulement en tant que certificat de fin d’études, mais aussi comme indicateur de performance du système éducatif.

Malgré les réformes successives visant à adapter ce diplôme aux réalités nationales et aux enjeux contemporains, le baccalauréat au Sénégal reste confronté à un taux de réussite relativement faible, 
oscillant autour de 40\%, loin des standards observés dans des pays comme la France ou le Canada où il s'oscille au tour de 99\% \cite{Mbaye2023}. 
Ce paradoxe entre le nombre croissant de candidats et la stagnation du taux de réussite interroge sur l’efficacité des politiques éducatives mises en œuvre.

Dans ce contexte, plusieurs réformes majeures ont été introduites : l’instauration du baccalauréat arabo-français en 2000, la création du bac arabe en 2013, 
et la transformation de la série G en série STEG (Sciences et Techniques de la Gestion) en 2019. 
Ces réformes traduisent la volonté des autorités d’élargir les opportunités d’accès à l’enseignement supérieur, de diversifier les profils de bacheliers, 
et d'améliorer l'adéquation entre formation et marché du travail. Ce mémoire se propose d’évaluer l’impact réel de ces transformations sur le taux de réussite au baccalauréat et sur le parcours universitaire des bacheliers, notamment à l'UCAD.

\section*{Problématique et Objectifs de l'étude}
\addcontentsline{toc}{section}{Problématique et Objectifs de l'étude}

Les réformes successives du baccalauréat sénégalais, notamment l’introduction du bac arabe en 2013 et de la série STEG en 2019,
ont été motivées par la volonté de moderniser le système éducatif, de mieux adapter l’offre de formation aux besoins socio-économiques du pays et de favoriser l’inclusion. 
Toutefois, leur impact réel reste encore peu documenté de manière rigoureuse.

Une question centrale se pose :

\textbf{Ces réformes ont-elles réellement contribué à améliorer les performances au baccalauréat et à faciliter le parcours universitaire des nouveaux bacheliers ?}

Plusieurs interrogations découlent de cette problématique :
\begin{itemize}
    \item Les réformes ont-elles eu un effet mesurable sur les taux de réussite au baccalauréat, globalement et par série ?
    \item Comment les nouvelles séries, comme le bac arabe ou la série STEG, se positionnent-elles en termes de performances au bac ?
    \item Comment les bacheliers issus des séries arabes évoluent-ils dans l’enseignement supérieur, notamment à l’UCAD ?
    \item Existe-t-il des disparités significatives entre les différentes séries du baccalauréat en termes de réussite et d’insertion à l'UCAD ?
\end{itemize}

À travers une analyse des données du baccalauréat (2006–2024) et des inscriptions à l’UCAD (2002–2024), ce travail poursuit plusieurs objectifs :
\begin{itemize}
    \item \textbf{Analyser l’évolution des taux de réussite au bac}, avec une attention particulière aux périodes post-réformes.
    \item \textbf{Étudier le parcours universitaire des bacheliers (suivi de cohortes)}, en identifiant les départements, filières et performances selon les séries d’origine.
\end{itemize}

Ce travail vise ainsi à combler un manque crucial d’évaluation quantitative des réformes éducatives,
en mobilisant des méthodes statistiques et des visualisations, pour offrir une lecture factuelle et visuelle des effets des réformes, afin d’éclairer la prise de décision dans le secteur éducatif.

\section*{Méthodologie et Organisation du document}
\addcontentsline{toc}{section}{Méthodologie et Organisation du document}

Cette étude repose sur une démarche méthodologique structurée, combinant analyse statistique et visualisation avancée. L’objectif est d’évaluer de manière quantitative et visuelle l’impact des réformes du baccalauréat sur la réussite scolaire et le parcours universitaire.
\newpage
La méthodologie adoptée comprend :

\begin{itemize}
    \item Une \textbf{revue des réformes éducatives} ayant marqué le baccalauréat sénégalais.
    \item L’exploitation de trois grandes bases de données : les \textbf{résultats du baccalauréat} (2006–2024), les \textbf{inscriptions universitaires à l’UCAD} (2002–2024) et les \textbf{résultats universitaires à l’UCAD} (2011–2024).
    \item Une \textbf{analyse Exploratoire}, appuyée par des représentations visuelles riches afin de faciliter la compréhension et l’interprétation des résultats.
    \item Un \textbf{suivi longitudinal des cohortes} de bacheliers à l’université, avec une attention particulière portée aux différences selon les séries d’origine.
    \item Une \textbf{restitution interactive} à l’aide de tableaux de bord dynamiques réalisés avec Power BI.
\end{itemize}


Le document est organisé en six chapitres principaux, suivis d'une section dédiée à la conclusion et aux recommandations, suivant la logique de cette démarche :

\begin{enumerate}
    \item \textbf{État de l’art} : Présentation des réformes du baccalauréat et revue des études antérieures sur le sujet.
    \item \textbf{Données et outils utilisés} : Description des sources de données et des outils mobilisés.
    \item \textbf{Nettoyage et préparation des données} : Traitement, fusion et structuration des données pour les rendre exploitables.
    \item \textbf{Analyse du taux de réussite au bac} : Analyse visuelle du taux de réussite au baccalauréat à travers des graphiques.
    \item \textbf{Analyse du parcours universitaire et suivi de cohortes} : visualisation du cheminement des bacheliers à l’UCAD.
    \item \textbf{Restitution interactive et visualisation} : Présentation des dashboard Power BI, incluant les filtres dynamiques, mesures, colonnes conditionnelles et visualisations interactives.
\end{enumerate}

\textbf{Conclusion et recommandations} : Bilan général de l’étude, limites rencontrées et propositions pour renforcer l’efficacité des politiques éducatives.